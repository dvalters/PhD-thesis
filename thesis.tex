% Delete the MSc option if you are doing a PhD, or replace it with MPhil
% for a Master of Philosophy thesis
%
% The 12pt option is required by the 2001/02 thesis regulations
% Last update 15th August 2007: new Abstract format and Copyright Statement

% Replace MSc with PhD for PhD theses
% Remove the twoside option for single-sided printing
\documentclass[12pt,oneside,PhD]{muthesis}

\usepackage{lineno}
\linenumbers

%these are now in the muthesis.cls file
%\usepackage[utf8]{inputenc}
%\usepackage{graphicx}
%\usepackage{amsmath}
%\usepackage{amsmath}
%\usepackage{amsfonts}
%\usepackage{amssymb}

\begin{document}
\title{Sensitivity of numerical landscape evolution models to the spatial resolution and complexity of precipitation}

\author{Declan A. Valters}
% Faculty of Life Sciences people should comment the next line out
\school{Earth, Atmospheric, and Environmental Sciences}
\faculty{Engineering and Physical Sciences}
\def\wordcount{xxxxx}

% Uncomment the line below to suppress the `List of Tables' page (optional)
%\tablespagefalse

% Uncomment the line below to suppress the `List of Figures' page (optional)
%\figurespagefalse

% Uncomment the line below to use a customised Declaration statement
%\def\declaration{All the work in this thesis has been sourced from Google.}

\beforeabstract

Write your abstract here: Remember, it must fit on this A4 page and should
describe contents of the thesis/dissertation. Here might also be a good place
to indicate what you have achieved in the thesis/dissertation and, in the
case of a PhD, what new results you have discovered. Note that for a PhD
single-spacing is used throughout the Abstract, including displayed equations
%\[
%e = mc^{2}
%\]
as for the above example.

\afterabstract

% The next part is optional; however it is a good place to thank your
% supervisor and the people responsible for providing computer support ;-)
\prefacesection{Layperson's abstract}
An optional section suggested by the UoM thesis preparation guide.

\prefacesection{Acknowledgements}
Acknowledgements...


% The next line is NOT optional and MUST appear
\afterpreface


% Finally, you can start writing about all the new theorems you have proved
% and all the new results that you have discovered

%=-=-=-=-=-=-=-=-=-=-=-=
% MOTIVATION, BACKGROUND
%=-=-=-=-=-=-=-=-=-=-=-=
\chapter{Introduction}
\textit{This chapter introduces the topic in a broad setting within the context of landscape evolution under different climatic conditions and meteorology (rainfall spatial patterns). The topic is discussed in its historic context as well briefly. It then goes on to clarify the topic of interest, and relates it to both topical research questions in landscape evolution theory and also the societal impact of intense, erosive rainfall events on the landscape.}

\chapter{Modelling landscape evolution}
\textit{This is a general overview of landscape evolution models and the key concepts and limitations they have. }

\textit{In the next chapter, Chapter 3, there is a more focused review of the specific area of Landscape evolution models that I am working on -- i.e. rainfall representation in LEMs. I only touch upon rainfall parametrisation here.}

This chapter could be based on the \textit{Geomorphological Techniques} book chapter (Valters, 2016).

\chapter{Rainfall representation in current landscape evolution models}
\chaptermark{Rainfall representation in current models}

\section{Introduction}
This chapter reviews how current landscape evolution models represent rainfall input into the landscape system. It is worth stating here what exactly is meant by rainfall input, in the context of the atmosphere--land-surface system as represented in numerical models. Perhaps equally as importantly, it is worth discussing what aspects of rainfall are \textit{not} represented at all in any numerical models of landscape evolution, to clarify how geomorphologists conceptually think of meteorological processes acting on the landscape.

For most purposes, rainfall input in landscape evolution models is simply the quantity of water added to a surface cell or node, or to the whole model domain. In practise, no numerical models represent rainfall in the sense of it actually falling from the sky and hitting the ground. While this may seem a somewhat trivial point, the impact of individual rain drops on the land surface is known to be a in important contributor to surface erosion. Rain-splash erosion, as it is termed, is a well-studied phenomenon [cite a review of rainsplash erosion, if there is one?]. The interaction of raindrops with the surface is complex; it depends on the size of raindrop, falling velocity, angle of attack, soil exposure, soil mineralogy, and cohesion of the soil surface. All of these factors could affect both erosion on the landscape hillslopes, as well as the route that water takes to runoff and reach the rivers, before fluvial erosion can happen.

If we briefly turn to physical analogue models of landscapes, rainfall representation implicitly accounts for some of the above factors in rainsplash erosion and runoff, because of the physical need to generate a rainfall source from above the model, such as through a fine-meshed sprinkler [CITE]. In fact, geomorphologists using physical analogue models of landscape evolution attempt a degree of rainfall realism by ensuring the raindrops they generate are reasonably well scaled to the size of their landscape analogue (Meyer, 1994). By contrast, numerical models of landscape evolution begin their representation of rainfall input at the surface -- in effect rainfall input in most landscape evolution models has nothing to do with \textit{falling} rain or its impact on the ground. Conceptually, rainfall input in numerical models is the amount of water that would be added at the surface from one or more (usually many more!) raindrops, once they have reached the ground. It ignores any effects from the physical collision raindrops make with the ground. This simple conceptual model of rainfall input is used throughout the rest of this chapter when referring to rainfall input in landscape evolution models.

\section{Simple models and proxies for rainfall variation}

\subsection{1D models}
Isolated aspects of landscape evolution and hydrology can be studied using 1D models of features such as hillslopes profiles and longitudinal river profiles, or storm hydrographs in the case of hydrology. Though the work in this thesis focuses on 2D models, it is useful to consider the work done by others invesigating the feedback from rainfall variability on 1D models of landscape evolution, before progressing to full 2- or 2.5D\footnote{Occasionally, the terms 2D and 2.5D are used interchangeably when referring to landscape evolution models, although in effect they both produce what looks like a `3D' terrain surface from their ouput. The `third' dimension (or extra 0.5D in 2.5D terminology) comes from the fact that the elevation variable can be used to reconstruct a 3D picture of the landscape based on the value for each grid cell or node. In practice, nearly all of the process models in landscape evolution models are 2D, e.g. water routing over the surface does not account for turbulent flow in x, y and z directions, such as in computational fluid dynamic models. Sediment transport does not account directly for 3D particle motion or collisions between particles. I use the term 2D landscape evolution model throughout the work.} models over an \(x,y\) model domain. 

Roe et al. (2002) modify a simple 1D model for river profile evolution (Seidl and Dietrich 1992; Howard et al., 1994; Whipple and Tucker, 1999) to incorporate a feedback for orographic precipitation based on changing elevation along a steepening river profile. Their precipitation feedback model accounts for two precipitation regimes: the first typical of midlatitude, shallower, and narrower mountain ranges such as the West coast of North America, and one for broader and taller ranges such as the Sierra Nevada, European Alps or the Southern Alps of New Zealand. The former represents rainfall patterns that are dominated by the prevailing upslope winds, increasing precipitation with distance upstream, whereas the latter represents environments where atmospheric moisture content exerts more control over precipitation, resulting in decreasing rainfall at higher elevations, and a rainfall shadow on the leeward side of the range. In a later work (Roe et al., 2003) the 1D model incorporating orographic rainfall feedback is extended to the 1D relief structure of mountain ranges. The maximum relief is found to be strongly dependent on the type of precipitation regime chosen - with the prevailing upslope wind regime favouring lower relief, symmetric mountain ranges, and the atmospheric moisture-limited regime favouring higher relief mountain ranges.

Further 1D models have been developed to determine the relative importance of rainfall variability compared to other boundary conditions, such as tectonic uplift or base level fall. The 1D river profile model of Wobus et al. (2009) uses a transport limited formulation of river profile evolution (Meyer-Peter and Muller, 1948) with a simple parameterisation of rainfall based on modifying the exponent to the discharge-area approximation given by:

\begin{equation}
q_w = k_qA^c
\end{equation}

where \(q_w\) is the water discharge, \(k_q\) a dimensional coefficient, \(A\) the contributing drainage area, and \(c\) the exponent that relates which portions of the drainage basin contribute to gathering precipitation and converting it to water discharge. A decrease in \(c\) represents a shift to more rainfall being gathered in the upper reaches of the stream. An increase in \(c\) represents rainfall being gathered in the lower reaches. The situation where \(c = 1\) implies rainfall input is equal along all sections of the river profile. The end result is perhaps intuitive -- more rainfall input in the upper reaches of the stream (decrease in \(c\) ) results in more incision in the headwaters. However, the study reveals a key difference in the way that climatic and tectonic signals propagate along a river channel. Numerical results show that rainfall-driven perturbations propagate from the channel head downstream, whereas tectonic perturbations invariably propagate from base-level upwards towards the channel head. The authors, however, reach this conclusion without simulating the scenario where there is more contributing rainfall from the lower reaches, i.e. the value of \(c\) is higher. Given the setting of the study though, (streams draining a mountain front) it is perhaps reasonable to assume an increasing precipitation gradient upstream towards the mountain range.

In the one-dimensional cases discussed, there is a key limitation, which is often acknowledged by the authors. Channels profiles in 1D form are modelled with out their tributary streams. The main stem of the channel is assumed to be representative of the entire catchment as a whole. This implies that tributary channels, and hillslopes feeding the main channel, experience the same precipitation patterns, or that differences between the main channel and its contributing water sources can be ignored. 

River channel profiles are not the only markers of landscape evolution, though they do dominate the range of 1D modelling studies investigating sensitivity to the spatial distribution of rainfall. Owen et al. (2010) address the sensitivity of hillslopes to average precipitation rates, although spatial variation of rainfall along hillslope profiles is not considered. The study reveals hillslopes are most sensitive to average precipitation rates when there is a lack of vegetation. Hillslope bedrock erosion decreases according to a power law as mean rainfall rates decrease, from semi-arid to hyperarid environments. In general though, the study of hillsope sensitivity to the spatial distribution of rainfall remains under-studied, particularly in the case of 1D profile evolution.

One-dimensional profile models are useful tools for exploring aspects of landscape evolution. By their definition though, they restrict studies of rainfall spatial variability to a single dimension along the landform profile. Rainfall spatial variability from tributary channels, or from runoff over hillslopes is lost, or `smeared-out' (Roe et al., 2002). The effects of water routing within a drainage network are also lost, and interesting relationships between rainfall distribution, river network connectivity and erosion are potentially overlooked. Complex parameterisations of rainfall production are often reduced to a single number or exponent in an equation describing the evolution of the landform profile of interest. Rainfall spatial patterns are often complex over correspondingly complex terrain, and only 2D models may suffice to fully explore the sensitivity of landscape process and form to rainfall spatial distribution.

\subsection{2D models}
Early 2D numerical models of landscape evolution were often driven by single process laws of fluvial incision, and the topography that resulted from them was a product of the parameters in the fluvial incision laws. Simple fluvial incision laws, implemented in 2D numerical models resulted in topography broadly similar to the fractal patterns of river networks observed in nature [CITE Ahnert/Turcotte], with the hillslope features between neighbouring river channels been formed by what was `left behind' from fluvial incision patterns. In other words, separate process laws were not implemented to describe the typically diffusive processes observed in hillslope formation. [Roerring, Hurst, citations from Hurst]

A typical form of the simple stream power law for fluvial incision takes the form:

\begin{equation}
E = KA^mS^n
\end{equation}

where \(K\) is termed the coefficient of erodibility, and is a catch-all term for climatic processes (amongst others) including the role of rainfall on the fluvial incision process. The K term itself could be considered a proxy for rainfall variation over time, assuming all other factors remained constant. [\textit{are there studies that do this, I thought there were somewhere...not sure now?}]

Another simple model is the excess shear stress model for fluvial incision, where the incision or erosion rate, \(E\) is given as a function of shear stress, \(\tau\) above a threshold level, \(\tau_c\):

\begin{equation}
E = k_e(\tau^a -  {\tau^a}_c)
\end{equation}

With this simple model of landscape evolution, one of the first studies to study the 2D evolution of topography under varying climatic conditions was that of Rinaldo (1993). The study implemented a cyclic variation through time on the parameter of critical shear stress, the threshold for erosion, \(\tau_c\). Since shear stresses driving incision are determined by river discharge, which in turn is controlled by rainfall input, the cyclical variation in critical shear stress, \(\tau_c\) can be used a proxy for temporal variations in rainfall over the catchment at geological timescales. When the value of \(\tau_c\) is low during the model this effectively represents a period of high rainfall intensity, and when \(\tau_c\) is high this represents a period of lower intensity rainfall (Rinaldo, 1993). In the resulting topographies from these simulations, drainage density and fractal dimension were shown to increase in response to a decrease in critical shear stress, or an increase in rainfall input over time, assuming other factors such as uplift remain constant.

Other studies to expand on:
\begin{itemize}
\item CHILD (Tucker) - Precipitation Stochastic Model.
\item Colberg and Anders (2003)
\item Solyom and Tucker (2004) - is this distributed or not? Probably not because rainfall-runoff is a parameterisation.
\item Solyom and Tucker (2007)
\end{itemize}

\section{Distributed models}
Distributed models\footnote{I borrow the term `distributed model' here from hydrological modellers.} are grid-cell based (or based on a grid of `nodes') and allow certain variables to vary spatially across the model domain, from cell-to-cell or node-to-node. The term is less frequently used with regards to landscape evolution modelling, but is useful to distinguish those models which represent a spatial variability in meteorological input from those that treat it through a proxy variable or another parameterisation. There are comparatively few landscape evolution models that allow spatially variable rainfall input to be distributed across the model domain, and some of the examples discussed here are from purely hydrological models. However, the principal of modelling spatially distributed rainfall remains the same and there are potential applications in hydrological modelling that can be extended to landscape evolution purposes. 

\subsection{Hydrological models}
In the world of hydrological modelling, distributed rainfall inputs are more commonplace. A range of meteorological input data sources have been used to drive distributed hydrological models. Three main sources of spatial rainfall data commonly used are dense-network rainfall-gauge data, precipitation radar, and precipitation outputs from numerical weather prediction models. Each one of these sources has a range of merits and demerits as a raw data source, but the discussion here focuses on their suitability as spatially heterogenous rainfall datasets for numerical landscape evolution models, rather than an appraisal of their relative accuracies in reporting precipitation distribution.

Precipitation data generated by numerical weather prediction models has been successfully used in distributed hydrological models to make hydrological forecasts, as well as to analyse historic flooding events. Hay et al. (2006) use the MM5 model (mesoscale meteorological model)\footnote{A precursor to the Weather Research and Forecasting model, WRF.} to generate gridded rainfall data over a five-year period.  The rainfall data is used to drive the PRMS distributed hydrological model -- the Precipitation Runoff Modelling System -- over a corresponding five-year period. The numerical weather prediction model is run at grid cell spacings of 20km, 5km, and 1.7km, the finest of which resolves individual valleys and massifs, and captures the resulting rainfall patterns over the catchment at high resolution. The study also compares the way that rainfall input zones in the hydrological model are represented. In the hydrological model, different zones of rainfall input can be defined along natural topographic boundaries, which are termed \textit{Hydrological Response Units}. These rainfall zone units tend to follow sub-catchment boundaries within the main catchment watershed. Alternatively, the catchment can be divided up more simply into rainfall input zones corresponding to a regularly spaced grid at a cell-spacing that matches the resolution of the input data.
In general, increasing rainfall input resolution in the Hay et al (2005) study results in a greater accuracy when compared with observed river discharge values. Using irregular-shaped hydrological response units based on natural sub-catchments, rather than a regular gridding of input data, results in better agreement with observation. However, as resolution increases towards the 1.7km grid-cell spacing, the difference seen from using irregular shaped hydrological response units and regular grids of comparable resolution decreases. 

A study that uses high resolution numerical weather prediction model data  to drive a hydrological model (Younger et al., 2007) tests the suitability of rainfall forescast data for making hydrological predictions and improving flood forecasting. High resolution (250m grid spacing) simulations using the United Kingdom Met Office Unified Model are used to generate input rainfall data to drive a TOPMODEL-based (Beven and Freer, 2001) hydrological model. The semi-distributed \textit{Dynamic-TOPMODEL} hydrological model groups topographically similar regions of the catchment and calculates runoff-predction for each of the these self-similar zones. The runoff calculation is then assigned to each node in that particular zone (see Beven, 2002, for a full explanation of the TOPMODEL concepts.) Computationally, this is more efficient than performing runoff calculations for every single grid cell in the catchment domain.
The Younger et al. (2007) study considers two events, a summer convective rainfall-event and a winter stratiform rainfall event. Although the hydrological simulation using the dense-network of rainfall gauge data produced outputs more closely matched to discharge observations, simulations with the NWP rainfall forecast also produce accurate results. The authors highlight the potential of using high-resolution rainfall forecast data to improve flood-forecasting in the future, giving greater prediction lead-in times compared to nowcasting from rainfall radar or real-time raingauge measurements. Rainfall data from numerical weather prediction models lends itself well to use as input data for hydrological modelling; it is typically written in a gridded data output format, and if the user has control over both the generation of the NWP output as well as the hydrological or landscape evolution model, generating compatible data formats can be more straightforward.

A consensus has yet to emerge on whether distributed hydrological models are sensitive to the spatial distribution of rainfall input. Nictoina et al. (2008), in a study that assesses rainfall resolution in distributed hydrological models, note that several studies are in disagreement, even when comparing catchments of similar sizes and in similar environments. In terms of the peak discharge and the time to the peak from the onset of heavy rainfall during a flood, modelling rainfall input as a spatially heterogeneous boundary condition appears to have little impact on the predicted hydrographs.  (Krajewski et al., 1991; Shah et al, 1996). It is noted that antecedent conditions may determine some of the relative sensitivity in catchment hydrological response (Shah et al., 1996), but only when initial water saturation levels are low. The work by Shah, and that of Segond et al., (2007) indicate that variability in runoff production mechanisms are the dominant control on runoff response. Whether variability in rainfall heterogeneity also contributes to the runoff response depends on antecedent conditions, as catchments may be able to dampen spatial heterogeneities in rainfall (Segond et al., 2007). In the simulations run by Nicotina et al. (2008), the source of rainfall data is from a network of rain gauges. Rainfall resolution is varied by first interpolating the rain gauge data with inverse weighted kriging method to 100m resolution. The 100m resolution data is then upscaled to coarser grid-sizes of 10km and 50km, giving three sets of simulations. Their study uses two catchments of 1560km\(^2\) and 8000km\(^2\) in area. The authors select catchments of relatively large size compared to previous studies. Their choice of larger catchments is based on one their hypotheses being that smaller catchments are closer in size mesoscale rainfall features, and therefore less likely to experience truly heterogeneous spatial rainfall patterns. The results of the Nicotina study show small differences between flood hydrograph peaks, which is more pronounced for the larger (8000km\(^2)\) catchment. A further set of simulations also compares a conservative upscaling of rainfall resolution to a non-conservative upscaling -- i.e. the total volume if rainfall is not necessarily the same post-upscaling. The non-conservative upscaled rainfall resolutions display a greater difference in maximum flood discharge over the three rainfall resolutions than the conservative upscaling method. The authors assert that catchments are more sensitive to the total volume of precipitation than its spatial heterogeneity, although this is perhaps to be expected if the non-conservatively upscaled experiments simply add more water to the catchment at coarser rainfall resolutions. The authors' further experiments with different runoff-generation mechanisms show a much more marked sensitivity in hydrograph response, compared to rainfall spatial heterogeneity. 

From a hydrological perspective, it would appear that getting the total rainfall volume and runoff-generating mechanisms accurately represented in a hydrological model are more important than the spatial pattern of rainfall (Gabellani et al., 2007; Nicotina et al., 2008). However, the approach of previous studies has been to focus primarily on the flood hydrograph during these simulations, which is essentially the water discharge modelled (or measured) at a single point at the catchment outlet. Very few studies, if any, have properly addressed the 2D spatial extent of floodwaters in response to spatially variable rainfall inputs over a catchment. It seems an odd omission to investigate a boundary condition that is by definition spatially heterogeneous over three dimensions (the areal spatial pattern of a rainstorm, as well as the storm depth or intensity), and then to reduce the output to a modelled parameter at a single \(x,y\) coordinate on the model domain. This could be remedied in future research projects.

Intuitively, one might expect that in a river catchment system with its well defined boundaries and singular output point, that any mass-conserving model would produce similar results given water inputs of equal volume (here, I am excluding the non-conserving rainfall upscaling method used by Nicotina at al., 2008). The details of interest may lie in what goes on inside the model domain, rather than what comes out the outlet point. Nevertheless, the work done by the hydrological modelling community has laid some of the foundations for using spatially variable rainfall data in 2D landscape evolution models. A range of data input sources, and interpolation methods that have been successful in hydrological modelling. Some of the basic findings will also help guide the research in the later chapters, and development of an existing landscape evolution model in Chapter 4.

\subsection{Landscape evolution models}

Few of the currently available numerical landscape evolution models explicitly allow the user to vary the spatial distribution of rainfall across the model domain (Valters, 2016). At longer timescales, it can be argued that spatial variation in climatic conditions such as rainfall will eventually be averaged out over centuries and millennia, in effect negating any variation in rainfall spatial patterns (Solyom and Tucker, 2007; Tucker 2010). However, this assumption only holds true if we believe that storm location and rainfall patterns bear no relation to the underlying topography of a landscape or river catchment. In other words, the assumption is that on the short term there is no orographic influence, and on the longer term, that there is no link between evolving topography and evolving weather patterns in a region. Only in recent years, and in a select few studies, have geomorphologists begun to question this assumption. As interest in this question has grown, models have evolved to accommodate this feature. At the short term end of the landscape modelling spectrum (days to centuries), the latest releases of the CAESAR-Lisflood model (Coulthard et al., 2014) now allow for spatially variable rainfall input data. 

\paragraph{Coulthard and Skinner (2016)}

In a sensitivity study that systematically varied the rainfall input data spatial resolution, Coulthard and Skinner (2016) assessed landscape evolution model sensitivity in terms of sediment and water flux, and the spatial distribution of erosion in a mid-sized upland catchment (415km\(^2\)). Rainfall input data was sourced from precipitation radar, and rainfall data resolution is varied at 5km, 10km, 20km resolutions, as well as a `lumped' input where rainfall is averaged spatially across the whole catchment. When the source data is upscaled to finer resolution, the total volume of rainfall is conserved (in contrast to the non-conserving upscaling methods used by Nicotina et al, 2008). The simulations are run with typical rainfall data that is extended over a 30 year period. Compared to the uniform (lumped) precipitation data, increasing the rainfall data grid resolution increases sediment flux from the catchment. In the case of the highest resolution rainfall simulation (5km), sediment flux increases by over 100\% compared to the uniform rainfall case. Coulthard and Skinner's study separates natural spatial variation in rainfall patterns by randomizing the rainfall cell `tiles' from the precipitation radar data, in an attempt to remove any effects from orography in the catchment. In essence, their study is focused solely on the effects of rainfall data resolution alone, rather than the spatial patterns of rainfall in nature, which are often influenced by topography. The rainfall field randomising technique minimise biases from naturally occurring organisation in storm cells and orographic rainfall enhancement. 

\paragraph{Von Ruette, et al (2014)}

So far in this chapter, the discussion has been on landscape evolution models and studies that focus on hydrological, fluvial, and hillslope erosional processes. Numerical models of whole-landscape evolution have a recognized bias towards temperate-humid landscapes (Pazzaglia, 2004; Tucker and Hancock, 2010; Valters, 2016) and tend to focus on a limited gamut of geomorphic processes: hydrology, fluvial erosion, hillslope evolution, and sediment transport. However, the sensitivity of other landscape processes may well be sensitive to the spatial distribution of rainfall over a landscape. Landsliding is an often overlooked, yet important process in landscape evolution and frequently omitted in numerical models (Tucker and Hancock, 2010; Valters, 2016). Von Ruette et al. (2014) investigate the sensitivity of shallow landslide initiation to the spatial distribution of rainfall in a catchment, using a physical based catchment-scale landscape evolution model designed specifically for investigating landslide triggering, the \emph{CHLT} model (von Ruette et al, 2013). In their modelling study, they examine the initiation of shallow landslides under spatially uniform rainfall and a coarse grid-based spatially variable rainfall input, from a real event occurring in 2002. The rainfall input data is a product of integrated rain gauge data and rainfall radar measurements. As the coarseness of the data is high relative to the size of the study catchment, the authors use an inverse distance weighting interpolation method\footnote{An interpolation that gives preferential weighting to points that are closer to each other. The measured values closest to the prediction point of interest have more weighting, which diminishes with distance from the point of prediction.} to downscale the data to a 2.5m grid cell size, the same resolution as the digital elevation model data used in the study. The authors generate a further set of simulations with a set of artificial rainfall input grids at 500m grid cell size. In the model of landslide initiation in the authors' model, the main sensitivity is the rainfall intensity and the infiltration capacity of the soil. If rainfall intensity is too high, water will runoff before it can fully infiltrate the soil; there exists a sweet-spot where rainfall intensities are low enough that the soil will become saturated more readily, and more landslides will be initiated. In the simulations run with equivalent rainfall intensities, spacial heterogeneity exerts some control over the distribution of landslides, as certain grid cells experience high rainfall rates, whereas others experience lower rainfall rates, closer to the rainfall rate `sweet-spot', and consequently more landslide initiation. The findings of the von Ruette (2014) study are complex; sensitivity of landsliding initiation to rainfall spatial heterogeneity is dependent on a number of other conditions such as soil moisture capacity, infiltration rate, rainfall rate, and rainfall intermittency. Rainfall spatial distribution in a catchment exerts a control on whether these conditions will be optimal for landslide intitiation, since it controls local rainfall intensities. Von Ruette et al. conclude that both the spatial distribution of landslides and the total number of landslides triggered are sensitive to the spatial distribution of rainfall in a catchment, assuming other conditions such as infiltration capacity are near-uniform across the catchment.

\subsubsection{Longer term landscape evolution}

\paragraph{Solyom and Tucker (2007)}
Landscape evolution sensitivity to rainfall detail over much longer timescales, on the order of 100kyrs and greater, has been explored to a limited extent by a few studies. Solyom and Tucker (2007) investigate how limited storm size relative to the size and shape of the drainage basin, effects the evolution of landscape topography. In their model, storm cells are represented as circular patterns with peak rainfall intensities at the centre of the circle, decaying exponentially from the centre:

\begin{equation}
I = I_0 \exp(-L_s/L_0)
\end{equation}

where I is he rainfall intensity at a given point in the storm cell, \(I_0\) is the rainfall intensity in the centre of the storm, \(L_s\) is the distance from the centre of the storm to a given point in the storm cell and \(L_0\) is a characteristic length scale associated with the spatial decline of rainfall intensity.

Orographic effects on rainfall enhancement are excluded in the model. In Solyom and Tucker's simulation, a set of idealised diamond-shaped catchments are varied in their elongation (length-width ratio), while being subjected to a steady non-uniform rainfall field described by the exponential decay function, centred at the middle of the diamond-shaped catchment. The exact implementation details in the model code is not revealed by the authors of the study. Their simulations reveal that in general non-uniform rainfall patterns introduce a catchment-shape sensitivity to rainfall-runoff production, which in theory should effect the size and distribution of geomorphic processes throughout the catchment as well. The authors do not present examples of topographies generated by the model, but instead show the total catchment discharge in non-dimensionalised form (\(Q_p/A*I_0\)) compared to non-dimensionalised catchment length (\(L/\sqrt{A}\), where A is the catchment area). Their simulations indicate that the greatest sensitivity occurs when the size of the storm decline rate \(L_0\) is about half of the catchment radius. Solyom and Tucker's interpretation of this is that if storm intensity declines very rapidly over space, i.e. the storm cell is small, then the majority of runoff production occurs in the vicinity of the storm cell, and is therefore insensitive to the shape of the catchment (assuming the storm falls near the centre of the catchment.) If the storm intensity decline rate is small relative to the scale of the catchment then in contrast the catchment is relatively insensitive to catchment shape.

\paragraph{Han and Gasparini (2015)}

A more explicit look at the way topography is influenced by spatial variation in rainfall patterns is found in the recent work of Han and Gasparini (2015). Building on earlier work by Roe et al. (2004), who found the geometry of river long profiles to exhibit sensitivity  to an orographic rainfall feedback mechanism, they explore the sensitivity of the whole landscape over a 2D domain. Modifying the CHILD landscape evolution model (Tucker et al., 2001), they develop a parameterisation scheme for orographic rainfall based on the model of  Smith and Barstad (2004). In their implementation of Smith and Barstad's model, the user controlled variables governing rainfall production are given in Table 3.1. The model offers considerable control over many meteorological variables determining orographic rainfall. In a series of simulations under differing rainfall conditions, the authors find only a slight sensitivity of the concavity of the main trunk channels under spatially variable rainfall. They conclude that channel concavity is not generally sensitive to to orographic rainfall patterns, in contrast to the 1D profile model of Roe et al. (2002) which showed much greater sensitivity. The more revealing topographic metrics were found in planform study -- both the hypsometric integral\footnote{A measure of the fraction of a catchment above a given elevation, describing the distribution of elevations over the catchment. See Brocklehurst and Whipple (2004); Cohen et al. (2008).} and the channel steepness index\footnote{A measure of channel steepness normalised to drainage area; See Wobus et al. (2006).} were found to be more strongly linked to the orographic rainfall gradient. 

\label{HanParameters}
\begin{table}
\begin{tabular}{|c|c|}
\hline 
\textbf{Parameter} & \textbf{Units}  \\ 
\hline 
Initial cloud water column density & kg m \(^{-2}\) \\ 
\hline 
Initial hydrometeor column density & kg m \(^{-2}\) \\ 
\hline 
Time constant for conversion from cloud water to hydrometeors & seconds \\ 
\hline 
Time constant for hydrometeor fallout & seconds \\ 
\hline 
Wind speed & m s \(^{-1}\) \\ 
\hline 
Mountain half width & metres \\ 
\hline 
\end{tabular} 
\caption{User defined parameters in Han and Gasparini's (2015) orographic rainfall model implemented in CHILD.}
\end{table}

In the model domain, rainfall input values for each node are now calculated individually, rather than the uniform rainfall field used in standard versions of CHILD. The calculation is based on a number of factors including the elevation of the current grid node, the direction of the prevailing wind, and factors relating to water content in the atmosphere. As the elevation of grid nodes can change as topography evolves throughout the simulation, and rainfall inputs depend on the elevation of each node, there is an explicit feedback mechanism between orographic precipitation and landscape evolution represented in the model. 

\subsection{Summary of current model capabilities}

The capabilities of landscape evolution models have evolved in tandem with research needs in a piecemeal fashion. As climate change has become an important factor in driving research needs and interests, landscape evolution models have evolved themselves to cater for a range of climatic parameterisations at a range of time scales. Two-dimensional\footnote{Or 2.5-dimensional, if the elevation variable is considered a limited 3rd dimension, in the sense that elevation can go up or down in landscape evolution models, though the underlying process representation remains restricted two-dimensions in the \(x,y\) plane. For example water flow and sediment transport is not fully realised in 3D in any current landscape evolution model.} numerical models are increasingly used for forecasting and predictive purposes, as well as just answering theoretical research driven questions. Despite their potential however, 2D models of landscape evolution are only beginning to be developed to allow detailed spatial variation in many of the climatic variables, such as rainfall. This is seen in the CHILD landscape evolution model work of Han and Gasparini (2015) as well of the development of CAESAR-Lisflood (Coulthard and Skinner, 2016) to simulate spatially variable rainfall input fields. 

Recent advances in landscape evolution modelling have coupled hydrological model components with the core erosional process modules to produce truly hydrodynamic models that do not assume steady state discharge. For example the CAESAR-Lisflood model (Coultard et al., 2013), Landlab modelling framework (Tucker et al., 2015), and tRIBS model [CITE] all contain forms of distributed hydrological models to simulate the transfer of water as well as sediment between grid cells or nodes. At longer timescales, the meteorological processes representing rainfall over a landscape have been parameterised, though the detail of these parameterisation schemes can be quite sophisticated (e.g. Han and Gasparini, 2015).

\section{Research needs for landscape evolution modelling}

The sensitivity of landscape evolutionary processes to the spatial details of climate and precipitation is still relatively unexplored. Though the subject is more advanced in purely hydrological studies (Krajewski et al., 1991; Smith et al. 2005; Segond et al. 2007; Nicotina et al. 2008), there is still a lack of agreement on when sensitivities to rainfall heterogeneities become most pronounced, given the dependence on other aspects of catchment hydrology. The role of runoff generating mechanisms, the influence of vegetation, the influence of groundwater routing pathways, are all affected by the spatial distribution of rainfall in a catchment, yet further investigations into these competing factors are required to reach more consensus among the hydrological community. Though some authors claim there is an insensitivity of hydrological processes to rainfall heterogeneity over a catchment (Krajewski et al. 1991; Smith et al. 2005), a key difference in landscape evolution modelling is that many erosional processes are threshold dependent. When rainfall is uniformly applied across a catchment model, the shear stresses generated by water runoff and river discharge tend to follow a uniform distribution as well. Findings by Coulthard and Skinner (2016) find a pronounced sensitivity to rainfall data resolution in term of sediment flux from a catchment (upto 100\% increases), in contrast to the relatively small differences observed in purely hydrological models (e.g. Nicotina et al. 2008). Apart from the Coulthard and Skinner (2016) paper, no other studies have been found that systematically explore landscape evolution model response to rainfall data resolution. Studies have yet to explore the effect of different spatial patterns of rainfall on the geomorphic impacts of single severe storms. 

With regards to data sources for rainfall input into landscape evolution models, the most typical source is rainfall gauge data, for single sites or sparse networks across a catchment. Rainfall radar has also been explored as a potential source offering higher spatial resolution that most rain gauge data typically available (Coulthard and Skinner, 2016). Other potential sources include output from numerical weather prediction (NWP) models, or the use of artificial weather generators. These two sources offer the potential to explore a variety of different spatial patterns of rainfall data, without having to source them directly from historic events. High resolution rainfall radar data only goes back [XX] number of years [CITE] for example. With methods using NWP models to simulate idealised weather conditions, or using weather generators, researchers have the potential to explore sensitivity to the spatial patterns of rainfall for a variety of meteorological conditions, and the potential to systematically explore different distributions of rainfall on landscape evolution. 

There is still a great deal of unexplored ground for developing landscape evolution models beyond their current capabilities.  Developments are needed to accommodate further types of spatially variable climatic input data and their interpolation (e.g. von Ruette et al, 2014; Coulthard and Skinner, 2016), to develop new feedback models between topography and rainfall generation (e.g. Han and Gasparini, 2015), new parameterisations of storm cell morphology (e.g. Solyom and Tucker, 2007), and to develop models to take advantage of high-performance computing facilities. %(Valters \& Coulthard, 2016/7). 

\textit{More on the research needs here...Perhaps an itemised summary of outstanding questions yet to be answered.}

\subsection{Technological advances}

Landscape evolution modellers have in general been reluctant to take advantage of emerging technology or high performance computing systems to explore bigger problems, or to explore uncertainty in model output through ensemble simulations. By way of contrast, in fields such as meteorology, mineralogy, particle physics, and engineering, the use of high-performance compute facilities is commonplace. In part, this is due to many problems in landscape evolution modelling stemming from a lack of agreement over geomorphic process laws. There is still considerable uncertainty over which geomorphic `laws' are best suited to represent certain natural processes, and the answer can be dependent on the environment being studied. As such, modelling simulations in landscape evolution have often focused on investigating the big-picture, broad-brushed questions about how landscapes evolve as a supplement to empirical field based studies. Geomorphologists, perhaps quite justifiably, have not yet required large-scale computing facilities used in other fields, for their questions can be answered satisfactorily with reduced complexity numerical models. This is especially true as a large body of numerical landscape evolution modelling is used in an exploratory manner (Tucker et al, 2010, some other citations tyo go here...and in this paragraph) -- geomorphologists have been accused by some (Hancock et al., 2003; Pelletier, 2015) of being satisfied simply if their modelled landscape "\textit{looks about right...}" The era of purely qualitative geomorphology has long since passed, and new quantitative methods that can be applied such as the use of topographic metrics should employed. Returning to the comparison with other fields and their use of high-performance computing (HPC), these fields often suffer uncertainty that geomorphology does in the choice of process law or parameterisation used in a numerical simulation. However, this does not stop them from judicious employment of HPC. In fact, one of the strengths of HPC facilities is the capability to assess many hundreds, if not thousands, of scenarios in ensemble simulations -- addressing the uncertainty in  process laws and model parameters that have been noted by others in the modelling field (Tucker and Hancock, 2010; Pelletier, 2015).  A drive towards making use of high performance computing technologies is needed in geomorphology.


%=-=-=-=-=-=-=-=-=-=
% WORK
%=-=-=-=-=-=-=-=-=-=
\chapter{Development of a numerical landscape evolution model for high-performance computing}
\chaptermark{A landscape evolution model for HPC}

\section{Introduction}
This chapter describes the development of a suitable numerical landscape evolution model for carrying out simulations of landscape evolution on the short term, on the order of hours to days. As reviewed in previous chapters, the current range of landscape evolution models available to the scientific community lack detailed representation of precipitation, especially regarding its spatial distribution. 

Two aims of this thesis have driven the need to develop or extend an existing numerical model to address the questions of landscape sensitivity to the details of precipitation. One is the wish to explore uncertainty and sensitivity in model output through an ensemble of catchment simulations to varying spatial patterns of rainfall. Though ensemble analysis can be done on a standard desktop computer, repeated numerical simulations done serially, one after the other, would time consuming. If each ensemble simulation could be done in parallel, at the same time, a considerable reduction in compute time could be achieved, given a cluster-type computer with sufficient resources. Since each would run the same program, but with a different set of parameters or input data, no modification to an existing landscape evolution model code would be necessary, so long as it was compatible with a supercomputing environment. This type of parallel problem is referred to as \textit{trivially} or \textit{embarrassingly} parallel [CITE]-- it requires no special endeavour other than ensuring the program is compatible with the intended computing platform to be used.

The second aim is a requirement to investigate the hydrogeomorphic response of the landscape at short timescales -- during the passage of a single rainstorm. As reviewed in Chapter 2, and in Tucker and Hancock (2010), Valters (2016), most landscape evolution models assume a hydrological steady state. A common assumption, for example, is that water discharge at any given point can be approximated as a function of the upstream drainage area at that point. Given that the aim of the research is to look at spatial patterns of rainfall relative to a catchment, the assumption that all areas upstream of any point are wetted by rainfall can no longer hold true. Similarly, runoff generation and river discharge can not be assumed to be uniform over the catchment area, given a spatially heterogenous rainfall input. Therefore, the numerical model should capture the dynamics of runoff generation and water flow within the catchment, even if only to a simplified degree. 

The research aims laid out in the introductory chapter include investigation of landscape evolution at a range of timescales, from single storms, to topographic evolution over millions of years potentially. Given the large discrepancy in timescales, a variety of modelling approaches will be needed. For short timescales, a hydrodynamic model would be ideal, as the water flux and sediment transport within a catchment can be explored in greater detail than a steady-state-hydrology landscape evolution model.

Are there currently available models that meet these criteria? The CAESAR-Lisflood model meets the requirement of being appropriate for  simulation of short timescales (hours to days), and also has a non-steady-state hydrological component, based on the LISFLOOD hydrological model (Bates et al., 2010). However, it lacks compatibility with most cluster-computer systems, being a Windows-based program. The program code for CAESAR-Lisflood is open-source, which would normally lend itself to recompilation for a linux-based cluster computer. However, the source code for CAESAR-Lisflood is C\# which is incompatible with most supercomputing services for a number of reasons: Firstly, compilers on supercomputers typically support only C/C++ or Fortran codes. Secondly, C\# codes are usually reliant on Windows operating system libraries\footnote{The .NET framework}. The typical supercomputing environment is Linux-based, and such libraries are not available. There is an open source implementation of C\# for Linux environments available\footnote{The \textit{Mono} project. The project is not yet fully compatible with all C\# source code, however.  In any case, further problems would have arisen trying to cross-compile suitable static binaries for the relevant cluster compute nodes, given that Mono is not generally available on HPC services.}, however early attempts in the project to compile CAESAR-Lisflood using the Linux version of these software libraries were unsuccessful. In its present form, CAESAR-Lisflood is unsuitable for running on most supercomputing services, though it would serve as a useful base for further development.

\subsection{Software design requirements}
To summarise the previous section, the criteria for the landscape evolution model required for this research are:

\begin{itemize}
\item Allow the spatial variation of precipitation and use of spatially variable rainfall data.
\item Be suitable for simulations at short timescales, to investigate landscape response during a single severe rainfall event. 
\item Adaptability to run ensemble simulations for sensitivity analysis, for example multiple simulations on a high-performance computing (HPC) facility, or similar cluster computer.
\item Compatibility with a typical Linux-based supercomputer environment.
\item Simulate a range of sediment transport and bedrock incision laws.
\end{itemize}

Since none of the existing landscape evolution models fully met the three key criteria above, a suitable model was developed instead. In particular, it was notable the none of the current models discussed in the previous chapters were particularly well suited to running on cluster computing facilities. The CHILD and CAESAR-Lisflood models came close to meeting these criteria, so it was decided to redevelop a model based on one of these existing code bases. As discussed in the previous chapter, CAESAR-Lisflood was particularly suitable to simulating short-term landscape evolution, so this model was used as a starting point. The model developed and described in this chapter is an evolutionary progression from CAESAR-Lisflood. Many of the algorithms are reused in the model's development and the flow structure within the program is similar in many respects. 

\section{Program description}
It was decided to implement the model in the C++ programming language\footnote{You might be wondering why I didn't use C, given its widely supported status in supercomputing applications. I wanted to make use of the LSDTopoTools package, written in C++, which has tools for manipulating raster data and performing topographic analysis. The resulting software described in this chapter is tightly integrated with LSDTopoTools. C++ is also increasingly well supported on supercomputing platforms.}, given the wide ranging support of compilers on supercomputing platforms, and the relative ease of translating some of the existing algorithms in CAESAR-Lisflood from C\# to C++\footnote{The syntax differences between the two languages are fairly minimal, compared to other widely-used language in numerical supercomputing, Fortran. Also, at the time, the author didn't know enough Fortran.}. 

In the following subsections the model framework and process representation modules are described. The modified model is termed the \textit{High-performance Architecture Independent Lisflood-CAESAR} model, or HAIL-CAESAR for short. 


\subsection{Spatial framework}
The model operates at the catchment-scale, and the landscape domain is defined by the hydrological unit of the river catchment, delineated by the watershed boundaries. As a pre-processing step, the catchment of interest must be extracted from digital terrain model (DTM) data as a separate step in the workflow. 

The model domain is discretised as a series of square-gridded cells. The basic cellular automaton framework of CAESAR-Lisflood is maintained in the model.  Every cell in the model domain grid has a series of associated landscape state variables (elevation, water depth, water velocity...), these landscape state variables are shown in figure [FIGURE]. In line with cellular automaton modelling theory, every cell has a set of neighbouring cells, which are used to calculate the next state of the cell at a specified time step. The model domain operates over a 2D surface, so each cell has four adjacent neighbour cells (referred to as `Manhattan' neighbours). Diagonally adjacent cells are not considered neighbours for the purpose of calculating fluxes between cells. The grid cell spacing is uniform across the model domain and there is no `adaptive remeshing' feature such as is found in the CHILD model (Tucker et al., 2001). 

\subsection{Process representation}

The main processes in the model are shown in [FIGURE]. In the majority of cases, the algorithms used are identical to the implementations described in Coulthard et al. (2013) and Bates et al. (2010). The key processes here are briefly reviewed, noting any implementation differences in the model




\section{An extended module for rainfall-runoff generation and interpolation}
%\chaptermark{A rainfall runoff generating module}

\textit{This section describes the rainfall interpolation and runoff generation modules written as part of the extensions to the CAESAR Lisflood model. It generates rainfall fields at the same resolution as the topographic data supplied (DEM), either from rainfall radar data supplied to it, or from an artificial storm generator, given some parameters for rainfall intensity, storm morphology etc.
}
\chapter{The hydrogeomorphic response of small catchments to rainfall radar resolution and patterns}
\chaptermark{Catchment sensitivity to rainfall resolution}
\section{Introduction}
Landscape evolution at the catchment scale is punctuated by intense erosive episodes driven by flood events (Wolman and Miller, 1960; Newson, 1980; Costa and O'Connor, 1995) interspersed with periods of relative calm and little geomorphic change, an idea that harks back to the early ideas of geological `catastrophism' (Cuvier, ?). It is these erosive events, driven by intense rainfall in temperate climates, separated by long periods of stasis, that cumulatively sculpt the landscape over geological time. The importance of these rare but formative events has been revisited by recent work such as Huang and Niemann (2006), looking at the long term implications of different geomorphically effective event discharges on fluvial incision; Gupta et al., (2007); Lamb and Fonstad (2010); and Baynes et al., (2015) where the amount of bedrock erosion during a single large flood event was quantified. Still, our understanding of catchment scale landscape evolution is far from complete - the role of individual events is highly variable. Anton (2014) report that rapid gorge formation can be driven primarily by small to moderate sized floods, rather than floods of extreme magnitude. Further, there was no observed relationship between flood magnitude and erosion rate. Turowski et al. (2011) report that streams within catchments can exhibit different behaviour in response to the same flood event -- some streams may erode during high flows, whereas others may deposit during high flows. During small--medium flows their respective behaviour is reversed. Wong et al. (2015) establish through numerical modelling that geomorphic changes in channel geometry during severe flood events are substantial enough to change hydrological response of a river catchment. Catchment-scale erosional dynamics are complex, and except in the simplest cases depend on other forcings other than the magnitude alone of single flood events.  The understanding of hydrogeomorphic processes during single storm events is not only important for the long-term evolution of landscapes, but also for prediction of how catchments will respond to changing hydro-meteorological conditions that may accompany climate change (Kendon et al., 2014).

The focus in this paper is to quantify the sensitivity of catchment-scale erosional processes to the spatial distribution of rainfall during flood events. The assumption of uniform rainfall over a river catchment is argued to hold true for small catchments (Solyom and Tucker 2004; Tucker 2010), but even over small areas, mesoscale rainfall features, such as localized convective storm cells, can result in spatially and temporally uneven input of precipitation into the catchment. In the case of intense convective precipitation, individual storm cells can be as small as 10km$^2$ in areal extent (Weisman and Klemp, 1986; Von Hardenberg et al., 2003). Over larger catchments, or those with steep topographic gradients, precipitation is almost certain to vary spatially, due to orographic enhancement of rainfall (Roe, 2005). As such, rainfall-runoff generation, local river flow, and erosion rates may vary considerably within individual drainage basins. 

Patterns of rainfall distribution across a catchment can affect hydrograph response, including the peak discharge and local water levels (Nicotina et al., 2008). As many geomorphic processes are threshold dependent (Schumm, 1979), such as fluvial incision into bedrock (Sklar and Dietrich, 2001; Snyder et al., 2003), there is potential for the spatial distribution of rainfall to control local erosion rates within a catchment. Non-linearity in geomorphic process laws (e.g. Coulthard et al., 1998; Phillips, 2003; Coulthard and Van de Wiel, 2007) should dictate that catchments are also geomorphically sensitive to the spatial distribution of rainfall. 

Numerical models of landscape evolution usually omit a realistic distribution of rainfall input in favour of uniform, homogenised precipitation across the landscape. When precipitation is `lumped', either spatially or temporally in a catchment, local minima and maxima of precipitation are lost, and with discharge being a function of rainfall rate, this uncertainty propagates through to local discharges and erosion rates. The uncertainty in erosion rates is potentially exacerbated by the non-linearity and threshold dependence of erosive processes. The variability of precipitation is considered in many cases to be as important as total precipitation amount in determining erosional effectiveness (Tucker and Bras, 2000; Tucker, 2010). What is currently lacking in landscape evolution studies is a fuller understanding of how landscapes erode during individual storms, and in particular how erosional processes are sensitive to the details of precipitation across a catchment. 

In numerical models of landscape evolution, resolving the precise temporal and spatial details of rain storms and the hydrological response is often computationally prohibitive, especially over long timescales, and as such modellers have taken to using simpler parametrisiations of storm characteristics, such as using simple stochastic models to generate rainfall inputs and rainfall timeseries (Eagleson, 1978; Tucker et al., 2001). In studies of long term landscape evolution, the sensitivity of landscapes to the spatial distribution of rainfall has been investigated to some extent -- particularly the imprint of orographic precipitation on landscapes (e.g. Roe 2002; Anders 2008; Han and Gasparini, 2015). Over the medium term, a study that systematically varied the resolution of rainfall input data in a decadal-scale catchment simulation (Coulthard and Skinner, 2016), local as well as catchment-wide sediment yields were predicted to increase by orders of magnitude as rainfall resolution increased. The study looked at the effects of rainfall data resolution alone, and not at the spatial distribution of rainfall itself, which was randomised for the simulations. 

%The intermediate stage between rainfall episode and geomorphic process is the hydrological response of the landscape. This is determined by a range of factors, includng antecedent conditions, bedrock and soil properties, vegetative cover, catchment morphology and the distribution of the rainfall across the catchment. Again, in numerical models of long term landscape evolution, these processes are parameterised for computational efficiency, generally through the selection of an appropriate rainfall-runoff model for the catchment (Beven). Most models of longer term landscape evolution assume a hydrological steady-state during each erosive event, and erosion is calculated as a function of peak or total discharge during a storm. Notable exceptions include the work of Solyom and Tucker (2004), where a parameterised non-steady state rainfall-runoff model is proposed to explain landscape evolution under condtions of short storm duration realtive to runoff time. 
%Talk about some of the purely hydrologic studies that have used spatially varible dsitributed rainfall.

In contrast to previous studies, this paper looks at the effects of individual events and the distribution of rainfall during those events on catchment hydrogeomorphic response. The study investigates the sensitivity of catchment-scale erosion to the spatial details of severe rain storms -- the agents of long term landscape evolution. Landscape response is investigated using a numerical landscape evolution model that incorporates a dynamic (non steady-state) water-routing component and a range of fluvial incision and sediment transport laws. A series of model experiments is presented to test how sensitive real landscapes are to the catchment-scale details of precipitation during intense rainfall events. The simulations are each based on selected severe storms in Great Britain occurring in the past decade, which left significant flooding, damage, and geomorphic change in their wake.

The following questions are explored through the use of numerical modelling simulations;

\begin{itemize}
\item Are fluvial erosion and sediment transport processes sensitive to the details of precipitation at the catchment scale during single storm events?
\item Does the choice of conceptual model representing the catchment influence sensitivity to rainfall patterns? 
\item What is the link between the spatial pattern of rainfall and spatial pattern of erosion and deposition in each storm--flood event?
\item What are the implications of this for longer term landscape evolution? 
\end{itemize}

%%%%%%%%%%%%%%%%
\section{Theory}
%%%%%%%%%%%%%%%%
In our conceptual model of landscape evolution, fluvial processes -- erosion and deposition of sediment and bedrock by flowing water -- are assumed to be the dominant geomorphic processes at work. The numerical model used to simulate these processes uses established hydrological and geomorphic process laws, which are breifly reviewed here in the following sections.

\subsection{Rainfall-runoff and flow routing}
From rainfall input, runoff is calculated using an adaptation of the Beven and Kirby (1979) TOPMODEL. Total surface and subsurface discharge is given by:

\begin{equation}
Q_{tot} = \frac{m}{T}\log \left( \frac{(r - j_t) + j_t \exp \left( \frac{rT}{m} \right) }{r} \right)
\end{equation}

where \(T\) is the time step in seconds, \(r\) is the rainfall rate, \(j_t\) is a function that describes soil moisture store, and \(m\) is a parameter that controls the rise and fall of this soil moisture store in \(j_t\). These adapted TOPMODEL equations are given fully in Coulthard (2002), equations (1) and (2).

The amount of water partitioned between surface and subsurface flow is determined by a simple infiltration threshold, given by:

\begin{equation}
I_t = KS(Dx)^2
\end{equation}

where \(K\) is hydraulic conductivity, \(S\) is the slope, and \(Dx\) is the width of the grid cell or horizontal grid spacing. The infiltration threshold is subtracted from \(Q_{tot}\) to give the portion of water routed over the surface.

Surface water and channel flow is an important driver in catchment scale erosional processes. The amount and velocity of water flow is a variable in both the sediment transport and bedrock erosion laws. The water flow equations are based on a simplified form of the shallow water flow equations, a simplification first derived by Bates (2010) and incorporated into the landscape evolution model by Coulthard et al (2013). The flow between cells is calculated by:

\begin{equation}
Q = \frac{q - g h_{flow} \Delta T \frac{\Delta (h+z) }{\Delta x}}{1 + g h_{flow} \Delta t n^2 |q| / h_{flow}^{10/3}} \Delta x
\end{equation}

where \(q\) is the water flux between cells from the previous iteration, \(g\) is acceleration due to gravity, \(h_{flow}\) is the maximum depth of flow between cells (m), \(t\) is time (s), \(h\) is depth of water, \(z\) is elevation, \(x\) is the grid well width, and \(n\) is Manning's roughness coefficient. The full implementation details are given in Coulthard (2013), and the derivation from the shallow water equations is given in Bates (2010).

\subsection{Sediment transport}
Transport of loose sediment is governed by the \citet{Wilcock2003} sediment transport model. The Wilcock and Crowe model represents transport of mixed sand/gravel fractions based on the surface sediment composition. The rate of sediment transport, \(q_i\), is given as:

\begin{equation}
q_i = \frac{F_i {U_*}^3 {W_i}^*}{(s -1) g}
\end{equation}

where \(F_i\) is the fractional volume of sediment, for a given sediment fraction, \(i\), \(U^*\) is the shear velocity, \(s\) is the ratio of sediment to water density. \({W_i}^*\) is a function relating fractional transport rate to total transport rate (see \citet{Wilcock2003} for a full derivation of this equation). The usage of this sediment transport model is extrapolated here to account for finer particles such as silts \citep{Vandewiel2007}, as well as the sand-gravel mixture it was originally designed for.

\subsection{Bedrock incision}
\label{bedrock_model}
A simple model of bedrock incision based on the excess shear stress model (Citations) is implemented in the numerical model. The rate of bedrock incision is determined by the amount of shear stress acting on the bedrock, above a threshold level of stress required to initiate substrate removal (e.g. \citet{Snyder2003}. When bedrock material is removed, it is distributed amongst the sediment fractions according to the fractional proportions set by the user. The rate of bedrock erosion according to the excess shear stress model is given by:

\begin{equation}
\varepsilon = k_e(\tau_b - \tau_c)^{P_b}
\end{equation}

where \(k_e\) is the bedrock erodibility coefficient, \(\tau_b\) is the basal shear stress on the channel bed, \(\tau_c\), is the critical shear stress threshold, \(P_b\) is the shear stress exponent. (Cite Howard or Whipple?)

%%%%%%%%%%%%%%%%%%%%%%%%%%%%%
\section{Experimental Design}
%%%%%%%%%%%%%%%%%%%%%%%%%%%%%
Three upland catchments in the UK were selected to represent a range of catchment sizes and shapes. The catchments were also chosen on the basis that they had experienced a severe rain storm which could be used as a basis for the experiments, such that it could be considered `extreme' in the typical return period of flooding events for each particular catchment. Peak discharges for each of the following flood events exceed the 99th percentile for their respective catchments. The catchments and respective severe rain events chosen were located in: Ryedale, North Yorkshire, 2005; Plynlimon, Mid-Wales, 2012; and Boscastle, Cornwall, 2004. An overview map of their locations is given in Figure \ref{overview_fig}. A table (Table \ref{met_setting}) summarises the key features of each catchment and associated storm.

\linespread{1.3}
\begin{table}
\resizebox{\textwidth}{!}
{%
\begin{tabular}{l c c c} \hline

Catchment Name& \textbf{Eden} 			& \textbf{Ryedale} &  \textbf{Valency} \\ \hline
Catchment Area & 2286km$^2$ 			& 270km$^2$ 				& 18km$^2$ \\ 
Catchment Type & Upland-Lowland & Upland, Moor/Peaty & Upland, Pasture \\ 
Storm Date	 		&  2005-01-07	 & 2005-06-19 	& 2004-08-16 \\ 
Peak Rainfall	 	& & & \\
Peak Discharge	 	& & & \\ 
Meteorological Setting	& 	& Split-front, convective system & Quasi-stationary convective system \\ 
Return Period 		& (tbc)						 & (tbc) 							& 1/200yr \\ \hline
\end{tabular}
}
\caption{Table showing matrix of experiments carried out for each catchment}
\label{met_setting}
\end{table}

\subsection{Meteorological setting}

\subsubsection{Boscastle, Cornwall storm 2004}
The Boscastle storm took place on the 16th August 2004 leading to flooding within the River Valency catchment and the village of Boscastle. The extreme rainfall accumulations of up to 200 mm in the upper Valency catchment resulted from prolonged rainfall between the hours of 1200 -- 1600 UTC. Rainfall rates were thought to have reached almost 400 mm hr\(^-1\) \citep{Golding2006}, after correcting for under-reporting from rain gauges in the vicinity of the catchment. (Burt, 2006, same issue).

The meteorological conditions that enabled such prolonged heavy rainfall were a combination of large-scale synoptic conditions moving in from the Atlantic, with moist lower atmospheric layers readily forming convective cloud. Repeated initiation of convection along the north Cornish coast lead to what appeared to be relative stationary convective cells over the Valency catchment. Later authors refer to this type of convective storm as a 'Boscastle-type' or quasi-stationary convective storm (cite the reading person).

\subsubsection{Ryedale, North York Moors storm 2005}
The Ryedale storm occurred on 19 June 2005. Intense rainfall throughout the afternoon lead to total accumulated rainfall amounts of up to 89mm in the Rydale valley, between the hours of 1400 -- 1800 UTC. Peak instantaneous rainfall rates were estimated to have reached 32.5mm hr\(^-1\) during the storm (Golding et al., 2005). The antecedent conditions had been dry for a prolonged spell, leading to cracking of the surface peat in the higher elevations of the catchment.

The meteorological conditions leading to such heavy rainfall was a combination of a cold, upper-level air mass advecting over a warm moist boundary layer, leading to unstable conditions that enabled a convective thunderstorm to develop in the late afternoon. The instability was enhanced by a split-frontal system. [More? Too much met here?]. The conditions let to a particularly high amount of precipitable water present in the atmosphere which was subsequently washed out into the landscape during intense rainfall. 

\subsubsection{Eden Valley, Carlisle, 2005}

\subsection{Numerical model set-up}
The landscape evolution model developed in Chapter 3 (Working name: HAIL-CAESAR) is used to carry out numerical simulations based on the three catchments and corresponding storm events. HAIL-CAESAR is a cellular automaton landscape evolution model based on the CAESAR-Lisflood model (Coulthard et al., 2013). The HAIL-CAESAR model simulates bedrock erosion according to a simple bedrock incision law based on critical shear stress. The equation describing the bedrock erosion model is described in section \ref{bedrock_model}. The bedrock incision model is used in two of the three sets of simulations. The model also interpolates and downscales rainfall input data to higher resolutions and this feature is used in the group of simulations with the 5m interpolated radar rainfall data. 

\subsubsection{Erosion model}
In order to address the uncertainty in choosing which erosion model applies for each catchment (Section \ref{theory}), three variations of model set-up are used, with each one representing a different conceptual model of fluvial incision and sediment transport. These include: i) a purely sediment transport-limited model, ii) a detachment-limited bedrock incision model, and iii) a hybrid model incorporating sediment transport and bedrock incision. The equations describing the transport-limited and detachment-limited models are discussed in Section \ref{theory}.

\paragraph*{Hybrid model}
The hybrid model assumes a limited-depth sediment layer, overlying a bedrock layer. Figure \ref{hybrid_model} shows a typical cross section through a typical valley in the hybrid model set-up. In the initial model state (before the spin-up period), a channel is 'burnt-in' to the sediment-layer. Whenever bedrock becomes exposed during the hybrid simulation, the simple detachment-limited erosion law is applied. Material removed from the bedrock layer is then apportioned between the various sediment fractions. At all other times, the sediment transport law applies to the sediment layer. 

\subsubsection{Rainfall spatial resolution}
In order to assess the sensitivity of each erosional model to the spatial details of precipitation, three different spatial resolutions of rainfall input are used in each simulation. All three are based on the same original rainfall source data - the UK NIMROD radar data product. Only the spatial distribution and resolution of rainfall is assessed in this study -- other studies have previously investigated the effects of the temporal resolution of rainfall data on discharge and erosion rates (e.g Nicotina et al., 2008; Coulthard and Skinner, 2015; Coulthard, 2013b). Three levels of rainfall detail are used: 

\begin{itemize}
\item Uniform or 'lumped' precipitation -- radar-derived rainfall rates across the catchment are spatially-averaged to produce a basin-wide average rainfall rate.  
\item Gridded rainfall input. The rainfall is input from a overlying gridded mesh of raincells, at the same resolution as the radar product (1km).
\item Interpolated rainfall input. The radar data is interpolated to the same resolution as the topography grid (i.e. 50m). \textit{(Interpolation method TBC - but see study by Tait et al (2006) and perhaps implement their method) }
\end{itemize}

The reason for running a simulation with an interpolated rainfall data set is to reduce the effect of harsh gradients between adjacent cells, as is sometimes apparent when using the rainfall data at its native resolution of 1km. Figure \ref{rainfall_input} gives an indicative illustration of this. A matrix of experiments is shown in Table \ref{model_setting}.

\subsubsection{Model spin-up}

The HAIL-CAESAR model (Valters ?\& Coulthard?, 2016/7) initialises the model domain with a uniform distribution of sediment grain sizes across the catchment. This is physically unrealistic, so the model domain is `spun-up' for a simulated time of 1000 days using typical rainfall data for each catchment. This ensures a heterogeneous distribution of sediment throughout the catchment prior to the detailed storm simulations. 

%%%%%%%%%%%%%%%%%
\section{Results}
%%%%%%%%%%%%%%%%%

\textit{(Probably separate sections for each)}

\textit{I intend to discuss the spatial differences in erosion, as well as any differences in basin-wide average erosion rates, and explain these differences by referring back to the Theory section.}

\subsection{Effect of rainfall detail on discharge and erosion}
The discussion will be aided by figures showing (for each of the three rainfall input variations for each catchment):

\begin{itemize}
\item Total accumulated rainfall maps for each storm (Figure \ref{ryedale}). 
\item Profiles of erosion along main channels in each catchment (Figure \ref{bigfig}).
\item Plots of the hydrographs and sediment yields for each storm (Figure \ref{bigfig}).
\item 2D Planform maps of distribution of erosion (and deposition if applicable).
\end{itemize}

\textit{Indicative figures. Note these will change in the final version as I have decided to re-run some simulations after tweaking the model set-up.}

\subsection{Implications for longer-term landscape evolution}
\textit{Some discussion on how these results scale-up to longer term landscape evolution. I.e. How many storms of similar magnitude would be needed to reach longer term erosion rates? Does this correspond to known longer term erosion rates of similar upland landscapes?}

%%%%%%%%%%%%%%%%%%%%%
\section{Conclusions}  %% \conclusions[modified heading if necessary]
%%%%%%%%%%%%%%%%%%%%%
Text.



\section{Fixed parameters}

\textit{A table showing the other parameters used in the simulations (All of which remain fixed for each simulation)}

The following table lists the parameters that were held constant for all simulations.

\chapter{Sensitivity of landscape evolution to the details of precipitation patterns using NWP model data}

\chapter{A spatially-limited storm generation model for long-term landscape evolution modelling}
\chaptermark{Storm morphology controls on topography}

\section{Intro}
In landscapes where fluvial processes are the dominant mechanism of sediment erosion and transport, several models of fluvial incision have been proposed that parameterise hydro-meteorological conditions -- the transfer of water from the lower atmosphere to the land surface. Such models include the representation of rainfall input as discrete storm events, e.g. the Poisson pulse model of rainfall input (Tucker and Bras, 2000), models that incorporate the role of limited storm duration relative to runoff-time across the catchment (Solyom and Tucker, 2004), and the nature of orographic precipitation gradients the rainfall-runoff-erosion process (e.g. Anders and Roe, 2006; Han and Gasparini 2015). 


\section{Hypothesis (mathematical model)}

\textbf{In short}: Rate of fluvial incision is dependent storm size, depth, and position of storm cell relative to the catchment. This applies to large catchments or small storm cells, where the ratio of storm coverage to total catchment size is less than one. 

\subsection{Catchment hydrology with limited storm cell sizes}
For large catchments, or for small storm cells, the area of rainfall input, here termed the catchment wetted area, will be less than the total catchment drainage area. (See figure 1). This is denoted by the ratio: \(A_w/A\), where \(A_w\) is the wetted area. 

%\begin{figure}
%\label{stormcellbasin}
%%\includegraphics[width=\textwidth]{drawing.png}
%\caption{Two scenarios in the limited area storm model, both with storm cell areal coverage less than the total basin area. Storm wetted area \(A_{wa} \approx A_{wb}\). a) Storm cell centred far from catchment outlet point, ratio of wetted flow runoff length to max basin length, \(L_w/L\), is near 1. b) Storm cell centred close to outlet point, \(L_w\) is short relative to basin total length. The implications on peak discharge, and thus erosion rates, are discussed within the text.}
%\end{figure}

Given a catchment with a smaller contributing storm cell, there can be variability in the positioning of the storm relative to the catchment outlet point. This spatial variation in storm cell positioning will influence the storm hydrograph for each storm event. The total storm hydrograph time, \(T_h\) can be given by:

\begin{equation}
T_h = T_r + T_t
\end{equation}

Where \(T_r\) is the storm duration time, and \(T_t\) is the total runoff travel time from the most distant wetted point in the catchment to the outlet. The total runoff travel time is given by:

\begin{equation}
T_t = L_w/U_f
\end{equation}

where \( L_w \) is the longest wetted flow routing path in the catchment, \(U_f\) is the flow routing velocity, assumed to be approximately spatially constant. 

\subsection{Deriving an approximation for discharge in storm-size limited catchments} 
Solyom and Tucker (2004) note that similar derivations for peak discharge approximation are needed for environments where storm size is limited relative to catchment area, so I use their equations as a starting point. 

Starting with the simple case of Hortonian (infiltration excess) overland flow, the discharge at a point in the river channel can be given by:

\begin{equation}
Q = (R - I)A_w
\end{equation}
where R is rainfall rate, I is infiltration rate, and \(A_w\) is the upstream wetted drainage area. The total volume of water in a given storm is stated as:

\begin{equation}
V = (R - I)A_w T_r
\end{equation}

where \(T_r\) is the storm duration time. As per Solyom and Tucker (2004) the total flood hydrograph volume can be written as:

\begin{equation}
V = \int_{0}^{T_h} Q(t) dt
\end{equation}

The hydrograph can be non-dimensionalised by scaling with peak flow, \(Q_p\), and time can be normalised by the total flood hydrograph duration, \(T_h\) (after Willgose, 1989):

\begin{equation}
Q'(t') = Q(t)/Q_p
\end{equation}
where
\begin{equation}
t' = t/T_h
\end{equation}

Then the non-dimensionalised flood-hydrograph volume can be written as such:

\begin{equation}
V = Q_p T_h \int_{0}^{1} Q'(t') dt'
\end{equation}

Assuming constant rainfall, \(R\), and infiltration rates, \(I\), peak discharge can be written as a function of runoff rate, storm duration, storm wetted area, and wetted flow route length:


%\begin{equation}
\begin{align}
\label{peakdischarge}
Q_p &= \frac{V}{T_h \int_{0}^{1} Q'(t') dt'} \
    &= \frac{1}{T_h} \frac{(R-I) A_w T_r}{F_{hs}} \
    &= \frac{(R-I)A_w}{F_{hs}} \frac{T_r}{T_r + L_w/U_f} 
\end{align}
%\end{equation}

where \(F_{hs}\) is a hydrograph `shape-factor' equal to the integral in equation (8). \(F_{hs}\) goes to one for steady state run off conditions (i.e. a flat or rectangular hydrograph)
 
According to equation \ref{peakdischarge}, peak discharge will vary according to the catchment wetted area to wetted flow runoff length ratio, \(A_w/L_w\), and the ratio of storm duration, \(T_r\), to the total hydrograph duration. Also note that \(A_w\) and \(L_w\) are not independent of each other, and increasing \(A_w\) can increase \(L_w\). (This is not true in the Solyom and Tucker (2004) application of this equation as rainfall is assumed to spatially uniform over the total catchment area.)

\subsection{Assumptions}

\begin{itemize}
\item Storm cell is stationary (does not track across the basin for the duration of the storm)
\item Storm cell is a single cell. (i.e. not multiple cells scattered across the basin)
\item Flow routing velocity is uniform. (Not entirely true assumption, routing time is slower on hillslopes, but the effect can be ignored if drainage density is relatively uniform (Solyom and Tucker (2004)).
\item Rainfall rate and infiltration rate constant for duration of storm.
\item Simple infiltration-excess (Hortonian) hydrological state.

\end{itemize}
It may be possible to modify the model to account for one or more of these assumptions.


\subsection{Fluvial erosion}
(Deriving a similar expression for fluvial incision in a detachment-limited environment here, incorporating the above non-steady discharge approximations for limited storm-area cases) 

%The rate of incision at any given point in a fluvial, bedrock channel is typically given as a function of excess shear stress, that is to say a certain threshold of shear stress exerted by the fluid flow of a river must be exceeded to detach bedrock material (Howard and Kerby, xxxx):

%\begin{equation}
%\epsilon = K_e(\tau - {\tau}_c)^\gamma
%\end{equation}
%
%where shear stress, \({\tau}_c \) and critical shear stress, \( {\tau}_c \), are given by a function of the local discharge \(q\), channel gradient, \(S\) and a shear stress coefficient, \(K_t\).
%
%Traditional models of fluvial incision typically consider discharge at any given point on the landscape as a function of local channel gradient and upstream constributing drainage area, such as the simple stream power law equation for fluvial incision in bedrock channels;
%
%\begin{equation}
%E = 
%\end{equation}


\chapter{Co-evolution of rainfall patterns and landscapes}

\textit{It would be nice to look at this if there was time (there probably won't be though!) I had a rough framework sketched out and partially implemented for how this would be done using CHILD and WRF together. Maybe one for the future instead.
}
\chapter{Synthesis: Painting rainy landscapes with numbers}

And here is the final synthesis chapter bringing it all together...

\bibliographystyle{plainnat}
\bibliography{thesis.bib}

% Comment the following THREE lines if you do NOT have an Appendix
\appendix
\chapter{Code availability}

\chapter{Key code and algorithms for the cellular automaton LEM}

\chapter{Key components and algorithms for the additions to the CHILD model}

\chapter{Modifications made to the Weather Research and Forecasting model (WRF)}

%\chapter{A one-way coupling framework for the WRF-CHILD model}

\chapter{Cluster computing simulations: set-up, compilation, and scaling}

\chapter{Paper off-prints may also be attached with a traditional-format thesis}
.........

% If you need more than one appendix, then just use another \chapter command
%\chapter{Yet Another Appendix}

\end{document}
