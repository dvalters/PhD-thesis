\chapter{Conclusions}
\label{chapter_conclusion}
\chaptermark{Conclsuions}

The spatial resolution of rainfall input in a river catchment simulation has a demonstrable effect on both hydrological and sediment fluxes, which is seen in both catchment-wide fluxes (Figures \ref{fig_boscastle_hydrograph_ensemble}, \ref{fig_boscastle_sedigraph_ensemble}, \ref{fig_ryedale_hydrograph_ensemble}, \ref{fig_ryedale_sedigraph_ensemble}). Though the magnitude of the sensitivity varied between experiments, there was a general increase in water fluxes, and hence sediment flux, in simulations with gridded rainfall data used as the rainfall input, compared to simulations that used a single, uniform value for rainfall. Existing studies investigating landscape evolution model sensitivity to rainfall resolution \citep{coulthard2016sensitivity} have noted similar increases in water discharge and sediment yields when using spatially heterogeneous rainfall data. The results from the experiments presented in this chapter are broadly in agreement with the general findings of \citet{coulthard2016sensitivity} although the timescales involved in each study are of different magnitudes. The results presented here suggest that the effect of rainfall resolution sensitivity applies to hydrogeomorphic processes at the single-event time scale, as well as landscape evolution over periods of hundreds and thousands of years.

Increasing the resolution of rainfall input data may not be enough to observe sensitivity in smaller catchments, as rainfall features themselves may not exhibit the necessary heterogeneity in structure to benefit from being resolved at finer scale. Rain cells or bands equal to or greater in size than the catchment over which they rain upon, may well be homogeneous enough in spatial extent and rainfall rate that a `uniform' approximation of their rainfall rate is sufficiently precise enough to represent the rainfall rate at all points in the catchment. As seen in the Boscastle catchment simulations, using a detailed rainfall input data source did not notably alter the outcome of the hydrological predictions (Figures \ref{fig_boscastle_2dplan_flood_ensemble}, \ref{fig_ryedale_2dplan_flood_ensemble}). In the Ryedale catchment simulations, the hydrological predictions were notably different based on the choice of rainfall input data resolution, affecting both the time and magnitude of the resulting flood peak. 
%
Talk about size of convective cell features. What is typical storm cell size and heterogeneity (decay from centre?).
