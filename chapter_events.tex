\chapter{Severe storm and flood events in Cornwall and Northern England}
\label{chapter_events}
\chaptermark{Severe storms in Cornwall and northern England}

\section{Introduction}

\section{Meteorological setting}
Two upland catchments in the UK were selected to represent a range of catchment sizes and shapes. The catchments were also chosen on the basis that they had experienced a severe rain storm which could be used as a basis for the experiments, such that it could be considered `extreme' in the typical return period of flooding events for each particular catchment. Peak discharges for each of the following flood events exceed the 99th percentile for their respective catchments. The catchments and respective severe rain events chosen were located in: Ryedale, North Yorkshire, 2005; Eden, Cumbria 2012; and Boscastle, Cornwall, 2004. An overview map of their locations is given in Figure \ref{overview_fig}. A table (Table \ref{met_setting}) summarises the key features of each catchment and associated storm.

\linespread{1.5}
\begin{table}
\resizebox{\textwidth}{!}
{%
\begin{tabular}{l c c} \hline

Catchment Name 			& \textbf{Ryedale} &  \textbf{Valency} \\ \hline
Catchment Area   			& 270km$^2$ 				& 18km$^2$ \\ 
Catchment Type         & Upland, Moor/Peaty & Upland, Pasture \\ 
Storm Date	 		            & 2005-06-19 	& 2004-08-16 \\ 
Peak Rainfall	 (mm hr \(^{-1}\))  & 125  & c.400 \\
Peak Discharge	 	 & & \\ 
Meteorological Setting	 	& Split-front, convective system & Quasi-stationary convective system \\ 
3hr Rainfall Return Period 	 & 330yr (Wass et al. 2008)	& 1300yr (Burt, 2005) \\ \hline
\end{tabular}
}
\caption{Table showing key characteristics of each storm event.}
\label{met_setting}
\end{table}

\subsubsection{Boscastle, Cornwall storm 2004}
The Boscastle storm took place on the 16th August 2004 leading to flooding within the River Valency catchment and the village of Boscastle. In the preceding months March--June, the south-west region had been drier than usual, but during July the Valency catchment area experienced average rainfall conditions \citep{golding2005boscastle}. The estimated soil moisture deficit\footnote{The amount of water needed to bring the soil back to field capacity, i.e. a state where the soil is holding the maximum amount of water possible against gravity. \citep{beven2011rainfall}} in the area, which had been lower than average due to the dry antecedent conditions, decreased during the period of 1--16th August, due to the return to average rainfall conditions in that month. The soil moisture deficit during this period was estimated to have decreased from approximately 80--220mm to 40--180mm.

On the day of the storm, the extreme rainfall accumulations of up to 200 mm in the upper Valency catchment resulted from prolonged rainfall between the hours of 1200 -- 1600 UTC. Rainfall rates were thought to have reached almost 400 mm hr\(^{-1}\) \citep{golding2005boscastle}, after correcting for under-reporting from rain gauges in the vicinity of the catchment. (Burt, 2006).

The meteorological conditions that enabled such prolonged heavy rainfall were a combination of large-scale synoptic conditions moving in from the Atlantic, with moist lower atmospheric layers readily forming convective cloud. Repeated initiation of convection along the north Cornish coast lead to what appeared to be relative stationary convective cells over the Valency catchment. Later authors refer to this type of convective storm as a `Boscastle-type' or quasi-stationary convective storm \citep{warren2014boscastle}.



\subsubsection{Ryedale, North York Moors storm 2005}
The Ryedale storm occurred on 19 June 2005. Intense rainfall throughout the afternoon lead to total accumulated rainfall amounts of up to 89mm in the Ryedale valley, between the hours of 1400 -- 1800 UTC. Peak instantaneous rainfall rates were estimated to have been around 32.5mm hr\(^-1\) (Sibley et al., 2009) to 59.4 mm hr \(^{-1}\) (Hopkins et al. 2010), though one report states they reached as high as 125 mm hr \(^{-1}\) (Cinderley, 2005). The antecedent conditions had been dry for a prolonged spell, leading to cracking of the surface peat in the higher elevations of the catchment.

Antecedent conditions before the Ryedale storm of 2005 had been dry over a prolonged period over much of the region \citep{sibley2009analysis}. Soil moisture deficit was estimated to be around 60mm in the catchmnet, higher than usual due to the drier conditions in the preceding months \citep{wass2008investigation}. With a low soil moisture content, thinner soils in the upper reaches of the catchment would have been dry before the intense rainfall.

The meteorological conditions leading to such heavy rainfall was a combination of a cold, upper-level air mass advecting over a warm moist boundary layer, leading to unstable conditions that enabled a convective thunderstorm to develop in the late afternoon. The instability was enhanced by a split-frontal system. [More? Too much met here?]. The conditions let to a particularly high amount of precipitable water present in the atmosphere which was subsequently washed out into the landscape during intense rainfall. 



\section{Flooding and geomorphic change}

\subsubsection{Boscastle, Cornwall, 2004}
\subsubsection{Ryedale, England, 2005}

\section{Numerical simulation configuration}

\subsubsection{Model initialisation}

The simulations are set to begin 24 hours before the day of each intense rainfall event. The model is allowed to run for a further 24 hours after the event, giving a total simulation tme of 72 hours. The timing of the simulations is given in Table \ref{table_start_time_hydrog_sims}.

\begin{table}[htbp]
\begin{tabular}{l c  c}
\textbf{Event}  &   \textbf{Start time} &  \textbf{End time} \\
\hline 
Boscastle          &  2004-08-15 00:00  &  2004-08-17 23:59 \\
Ryedale             &  2005-06-18 00:00  &  2005-06-20 23:59 \\

\end{tabular}
\caption{Start and end times (UTC) for each 72 simulation. The major rainfall event occurs during the 2nd day in each simulation.}
\label{table_start_time_hydrog_sims}
\end{table}

\subsubsection{Erosion-enabled simulations}
A variety of erosion laws exist describing how landscapes erode from fluvial incision. The choice of erosion law for a given catchment depends on a variety of factors, such as the characteristic substrate material in the catchment -- is it predominantly loose sediment or cohesive, solid bedrock? In reality, landscapes are often a mixture of these two extremes, incorporating loose sediment on top of solid bedrock. Catchments also often exhibit a transition from rockier upland headwaters, to more thickly soil-mantled flood plains. In order to address the uncertainty in choosing which erosion model applies for each catchment (Section \ref{theory}), two erosion model end-members are used, with each one representing a different conceptual model of fluvial incision and sediment transport. These include: i) a purely sediment transport-limited model, ii) a detachment-limited bedrock incision model. The equations describing the transport-limited and detachment-limited models are discussed in Section \ref{theory}. Further models were considered, such as a hybrid transport-detachment limited erosion model, but it was deemed beyond the scope of this study, which is to focus on the sensitivity of rainfall resolution, rather than wide range of erosion and sediment transport models. 

A set of control simulations parameterising only runoff and surface water routing (no erosion taking place) were also carried out for comparison against the two erosion end-member simulations. (Table \ref{table_ensemble_experiments}.)

%\paragraph*{Hybrid model}
%The hybrid model assumes a limited-depth sediment layer, overlying a bedrock layer. Figure \ref{hybrid_model} shows a typical cross section through a typical valley in the hybrid model set-up. In the initial model state (before the spin-up period), a channel is 'burnt-in' to the sediment-layer. Whenever bedrock becomes exposed during the hybrid simulation, the simple detachment-limited erosion law is applied. Material removed from the bedrock layer is then apportioned between the various sediment fractions. At all other times, the sediment transport law applies to the sediment layer. 

\subsubsection{Rainfall spatial resolution}
In order to assess the sensitivity of each erosional model to the spatial details of precipitation, rainfall spatial parameterisation is alternated in each simulation between a spatially uniform rainfall input and a 1km gridded rainfall input. Both the spatially uniform and gridded inputs are based on the same original rainfall source data - the UK Nimrod 1km-composite radar data product \citep{metoffice2003nimrod}. To reduce the range of variables within the set of ensemble simulations, only the spatial distribution and resolution of rainfall is assessed in this study -- other studies have previously investigated the effects of the \textit{temporal} resolution of rainfall data on discharge and erosion rates \citep{nicotina2008impact,Coulthard2013,coulthard2016sensitivity}. In all experiments, the temporal resolution of experiments is maintained at 5 minute intervals. To summarise, the two types of rainfall spatial input used are: 

\begin{itemize}
\item Uniform or `lumped' precipitation: Radar-derived rainfall rates across the catchment are spatially-averaged to produce a basin-wide rainfall rate. In other words, every grid cell in the model domain receives the same rainfall rate at each rainfall data timestep.
\item Gridded rainfall input. The rainfall rate is input from a overlying gridded mesh of raincells, at the same resolution as the rainfall radar product (1km).
\end{itemize}



%\subsubsection{Model spin-up}
%
%The HAIL-CAESAR model (Valters ?\& Coulthard?, 2016/7) initialises the model domain with a uniform distribution of sediment grain sizes across the catchment. This is physically unrealistic, so the model domain is `spun-up' for a simulated time of 1000 days using typical rainfall data for each catchment. This ensures a heterogeneous distribution of sediment throughout the catchment prior to the detailed storm simulations. 

\section{Summary}