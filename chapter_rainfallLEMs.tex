\chapter{Rainfall representation in current landscape evolution models}
\label{RainfallInLEMs}
\chaptermark{Rainfall representation in current models}

\section{Introduction}
This chapter reviews how current landscape evolution models represent rainfall input into the landscape system. It is worth stating here what exactly is meant by rainfall input, in the context of the atmosphere--land-surface system as represented in numerical models. Perhaps equally as importantly, it is worth discussing what aspects of rainfall are \textit{not} represented at all in any numerical models of landscape evolution, to clarify how geomorphologists conceptually think of meteorological processes acting on the landscape.

For most purposes, rainfall input in landscape evolution models is simply the quantity of water added to a surface cell or node, or to the whole model domain. In practise, no numerical models represent rainfall in the sense of it actually falling from the sky and hitting the ground. While this may seem a somewhat trivial point, the impact of individual rain drops on the land surface is known to be a in important contributor to surface erosion. Rain-splash erosion, as it is termed, is a well-studied phenomenon [cite a review of rainsplash erosion, if there is one?]. The interaction of raindrops with the surface is complex; it depends on the size of raindrop, falling velocity, angle of attack, soil exposure, soil mineralogy, and cohesion of the soil surface. All of these factors could affect both erosion on the landscape hillslopes, as well as the route that water takes to runoff and reach the rivers, before fluvial erosion can happen.

If we briefly turn to physical analogue models of landscapes, rainfall representation implicitly accounts for some of the above factors in rainsplash erosion and runoff, because of the physical need to generate a rainfall source from above the model, such as through a fine-meshed sprinkler [CITE]. In fact, geomorphologists using physical analogue models of landscape evolution attempt a degree of rainfall realism by ensuring the raindrops they generate are reasonably well scaled to the size of their landscape analogue (Meyer, 1994). By contrast, numerical models of landscape evolution begin their representation of rainfall input at the surface -- in effect rainfall input in most landscape evolution models has nothing to do with \textit{falling} rain or its impact on the ground. Conceptually, rainfall input in numerical models is the amount of water that would be added at the surface from one or more (usually many more!) raindrops, once they have reached the ground. It ignores any effects from the physical collision raindrops make with the ground. This simple conceptual model of rainfall input is used throughout the rest of this chapter when referring to rainfall input in landscape evolution models.

\section{Simple models and proxies for rainfall variation}

\subsection{1D models}
Isolated aspects of landscape evolution and hydrology can be studied using 1D models of features such as hillslopes profiles and longitudinal river profiles, or storm hydrographs in the case of hydrology. Though the work in this thesis focuses on 2D models, it is useful to consider the work done by others invesigating the feedback from rainfall variability on 1D models of landscape evolution, before progressing to full 2- or 2.5D\footnote{Occasionally, the terms 2D and 2.5D are used interchangeably when referring to landscape evolution models, although in effect they both produce what looks like a `3D' terrain surface from their ouput. The `third' dimension (or extra 0.5D in 2.5D terminology) comes from the fact that the elevation variable can be used to reconstruct a 3D picture of the landscape based on the value for each grid cell or node. In practice, nearly all of the process models in landscape evolution models are 2D, e.g. water routing over the surface does not account for turbulent flow in x, y and z directions, such as in computational fluid dynamic models. Sediment transport does not account directly for 3D particle motion or collisions between particles. I use the term 2D landscape evolution model throughout the work.} models over an \(x,y\) model domain. 

Roe et al. (2002) modify a simple 1D model for river profile evolution (Seidl and Dietrich 1992; Howard et al., 1994; Whipple and Tucker, 1999) to incorporate a feedback for orographic precipitation based on changing elevation along a steepening river profile. Their precipitation feedback model accounts for two precipitation regimes: the first typical of midlatitude, shallower, and narrower mountain ranges such as the West coast of North America, and one for broader and taller ranges such as the Sierra Nevada, European Alps or the Southern Alps of New Zealand. The former represents rainfall patterns that are dominated by the prevailing upslope winds, increasing precipitation with distance upstream, whereas the latter represents environments where atmospheric moisture content exerts more control over precipitation, resulting in decreasing rainfall at higher elevations, and a rainfall shadow on the leeward side of the range. In a later work (Roe et al., 2003) the 1D model incorporating orographic rainfall feedback is extended to the 1D relief structure of mountain ranges. The maximum relief is found to be strongly dependent on the type of precipitation regime chosen - with the prevailing upslope wind regime favouring lower relief, symmetric mountain ranges, and the atmospheric moisture-limited regime favouring higher relief mountain ranges.

Further 1D models have been developed to determine the relative importance of rainfall variability compared to other boundary conditions, such as tectonic uplift or base level fall. The 1D river profile model of Wobus et al. (2009) uses a transport limited formulation of river profile evolution (Meyer-Peter and Muller, 1948) with a simple parameterisation of rainfall based on modifying the exponent to the discharge-area approximation given by:

\begin{equation}
q_w = k_qA^c
\end{equation}

where \(q_w\) is the water discharge, \(k_q\) a dimensional coefficient, \(A\) the contributing drainage area, and \(c\) the exponent that relates which portions of the drainage basin contribute to gathering precipitation and converting it to water discharge. A decrease in \(c\) represents a shift to more rainfall being gathered in the upper reaches of the stream. An increase in \(c\) represents rainfall being gathered in the lower reaches. The situation where \(c = 1\) implies rainfall input is equal along all sections of the river profile. The end result is perhaps intuitive -- more rainfall input in the upper reaches of the stream (decrease in \(c\) ) results in more incision in the headwaters. However, the study reveals a key difference in the way that climatic and tectonic signals propagate along a river channel. Numerical results show that rainfall-driven perturbations propagate from the channel head downstream, whereas tectonic perturbations invariably propagate from base-level upwards towards the channel head. The authors, however, reach this conclusion without simulating the scenario where there is more contributing rainfall from the lower reaches, i.e. the value of \(c\) is higher. Given the setting of the study though, (streams draining a mountain front) it is perhaps reasonable to assume an increasing precipitation gradient upstream towards the mountain range.

In the one-dimensional cases discussed, there is a key limitation, which is often acknowledged by the authors. Channels profiles in 1D form are modelled with out their tributary streams. The main stem of the channel is assumed to be representative of the entire catchment as a whole. This implies that tributary channels, and hillslopes feeding the main channel, experience the same precipitation patterns, or that differences between the main channel and its contributing water sources can be ignored. 

River channel profiles are not the only markers of landscape evolution, though they do dominate the range of 1D modelling studies investigating sensitivity to the spatial distribution of rainfall. Owen et al. (2010) address the sensitivity of hillslopes to average precipitation rates, although spatial variation of rainfall along hillslope profiles is not considered. The study reveals hillslopes are most sensitive to average precipitation rates when there is a lack of vegetation. Hillslope bedrock erosion decreases according to a power law as mean rainfall rates decrease, from semi-arid to hyperarid environments. In general though, the study of hillsope sensitivity to the spatial distribution of rainfall remains under-studied, particularly in the case of 1D profile evolution.

One-dimensional profile models are useful tools for exploring aspects of landscape evolution. By their definition though, they restrict studies of rainfall spatial variability to a single dimension along the landform profile. Rainfall spatial variability from tributary channels, or from runoff over hillslopes is lost, or `smeared-out' (Roe et al., 2002). The effects of water routing within a drainage network are also lost, and interesting relationships between rainfall distribution, river network connectivity and erosion are potentially overlooked. Complex parameterisations of rainfall production are often reduced to a single number or exponent in an equation describing the evolution of the landform profile of interest. Rainfall spatial patterns are often complex over correspondingly complex terrain, and only 2D models may suffice to fully explore the sensitivity of landscape process and form to rainfall spatial distribution.

\subsection{2D models}
Early 2D numerical models of landscape evolution were often driven by single process laws of fluvial incision, and the topography that resulted from them was a product of the parameters in the fluvial incision laws. Simple fluvial incision laws, implemented in 2D numerical models resulted in topography broadly similar to the fractal patterns of river networks observed in nature [CITE Ahnert/Turcotte], with the hillslope features between neighbouring river channels been formed by what was `left behind' from fluvial incision patterns. In other words, separate process laws were not implemented to describe the typically diffusive processes observed in hillslope formation. [Roerring, Hurst, citations from Hurst]

A typical form of the simple stream power law for fluvial incision takes the form:

\begin{equation}
E = KA^mS^n
\end{equation}

where \(K\) is termed the coefficient of erodibility, and is a catch-all term for climatic processes (amongst others) including the role of rainfall on the fluvial incision process. The K term itself could be considered a proxy for rainfall variation over time, assuming all other factors remained constant. [\textit{are there studies that do this, I thought there were somewhere...not sure now?}]

Another simple model is the excess shear stress model for fluvial incision, where the incision or erosion rate, \(E\) is given as a function of shear stress, \(\tau\) above a threshold level, \(\tau_c\):

\begin{equation}
E = k_e(\tau^a -  {\tau^a}_c)
\end{equation}

With this simple model of landscape evolution, one of the first studies to study the 2D evolution of topography under varying climatic conditions was that of Rinaldo (1993). The study implemented a cyclic variation through time on the parameter of critical shear stress, the threshold for erosion, \(\tau_c\). Since shear stresses driving incision are determined by river discharge, which in turn is controlled by rainfall input, the cyclical variation in critical shear stress, \(\tau_c\) can be used a proxy for temporal variations in rainfall over the catchment at geological timescales. When the value of \(\tau_c\) is low during the model this effectively represents a period of high rainfall intensity, and when \(\tau_c\) is high this represents a period of lower intensity rainfall (Rinaldo, 1993). In the resulting topographies from these simulations, drainage density and fractal dimension were shown to increase in response to a decrease in critical shear stress, or an increase in rainfall input over time, assuming other factors such as uplift remain constant.

Other studies to expand on:
\begin{itemize}
\item CHILD (Tucker) - Precipitation Stochastic Model.
\item Colberg and Anders (2003)
\item Solyom and Tucker (2004) - is this distributed or not? Probably not because rainfall-runoff is a parameterisation.
\item Solyom and Tucker (2007)
\end{itemize}

\section{Distributed models}
Distributed models\footnote{I borrow the term `distributed model' here from hydrological modellers.} are grid-cell based (or based on a grid of `nodes') and allow certain variables to vary spatially across the model domain, from cell-to-cell or node-to-node. The term is less frequently used with regards to landscape evolution modelling, but is useful to distinguish those models which represent a spatial variability in meteorological input from those that treat it through a proxy variable or another parameterisation. There are comparatively few landscape evolution models that allow spatially variable rainfall input to be distributed across the model domain, and some of the examples discussed here are from purely hydrological models. However, the principal of modelling spatially distributed rainfall remains the same and there are potential applications in hydrological modelling that can be extended to landscape evolution purposes. 

\subsection{Hydrological models}
In the world of hydrological modelling, distributed rainfall inputs are more commonplace. A range of meteorological input data sources have been used to drive distributed hydrological models. Three main sources of spatial rainfall data commonly used are dense-network rainfall-gauge data, precipitation radar, and precipitation outputs from numerical weather prediction models. Each one of these sources has a range of merits and demerits as a raw data source, but the discussion here focuses on their suitability as spatially heterogenous rainfall datasets for numerical landscape evolution models, rather than an appraisal of their relative accuracies in reporting precipitation distribution.

Precipitation data generated by numerical weather prediction models has been successfully used in distributed hydrological models to make hydrological forecasts, as well as to analyse historic flooding events. Hay et al. (2006) use the MM5 model (mesoscale meteorological model)\footnote{A precursor to the Weather Research and Forecasting model, WRF.} to generate gridded rainfall data over a five-year period.  The rainfall data is used to drive the PRMS distributed hydrological model -- the Precipitation Runoff Modelling System -- over a corresponding five-year period. The numerical weather prediction model is run at grid cell spacings of 20km, 5km, and 1.7km, the finest of which resolves individual valleys and massifs, and captures the resulting rainfall patterns over the catchment at high resolution. The study also compares the way that rainfall input zones in the hydrological model are represented. In the hydrological model, different zones of rainfall input can be defined along natural topographic boundaries, which are termed \textit{Hydrological Response Units}. These rainfall zone units tend to follow sub-catchment boundaries within the main catchment watershed. Alternatively, the catchment can be divided up more simply into rainfall input zones corresponding to a regularly spaced grid at a cell-spacing that matches the resolution of the input data.
In general, increasing rainfall input resolution in the Hay et al (2005) study results in a greater accuracy when compared with observed river discharge values. Using irregular-shaped hydrological response units based on natural sub-catchments, rather than a regular gridding of input data, results in better agreement with observation. However, as resolution increases towards the 1.7km grid-cell spacing, the difference seen from using irregular shaped hydrological response units and regular grids of comparable resolution decreases. 

A study that uses high resolution numerical weather prediction model data  to drive a hydrological model (Younger et al., 2007) tests the suitability of rainfall forescast data for making hydrological predictions and improving flood forecasting. High resolution (250m grid spacing) simulations using the United Kingdom Met Office Unified Model are used to generate input rainfall data to drive a TOPMODEL-based (Beven and Freer, 2001) hydrological model. The semi-distributed \textit{Dynamic-TOPMODEL} hydrological model groups topographically similar regions of the catchment and calculates runoff-predction for each of the these self-similar zones. The runoff calculation is then assigned to each node in that particular zone (see Beven, 2002, for a full explanation of the TOPMODEL concepts.) Computationally, this is more efficient than performing runoff calculations for every single grid cell in the catchment domain.
The Younger et al. (2007) study considers two events, a summer convective rainfall-event and a winter stratiform rainfall event. Although the hydrological simulation using the dense-network of rainfall gauge data produced outputs more closely matched to discharge observations, simulations with the NWP rainfall forecast also produce accurate results. The authors highlight the potential of using high-resolution rainfall forecast data to improve flood-forecasting in the future, giving greater prediction lead-in times compared to nowcasting from rainfall radar or real-time raingauge measurements. Rainfall data from numerical weather prediction models lends itself well to use as input data for hydrological modelling; it is typically written in a gridded data output format, and if the user has control over both the generation of the NWP output as well as the hydrological or landscape evolution model, generating compatible data formats can be more straightforward.

A consensus has yet to emerge on whether distributed hydrological models are sensitive to the spatial distribution of rainfall input. Nictoina et al. (2008), in a study that assesses rainfall resolution in distributed hydrological models, note that several studies are in disagreement, even when comparing catchments of similar sizes and in similar environments. In terms of the peak discharge and the time to the peak from the onset of heavy rainfall during a flood, modelling rainfall input as a spatially heterogeneous boundary condition appears to have little impact on the predicted hydrographs.  (Krajewski et al., 1991; Shah et al, 1996). It is noted that antecedent conditions may determine some of the relative sensitivity in catchment hydrological response (Shah et al., 1996), but only when initial water saturation levels are low. The work by Shah, and that of Segond et al., (2007) indicate that variability in runoff production mechanisms are the dominant control on runoff response. Whether variability in rainfall heterogeneity also contributes to the runoff response depends on antecedent conditions, as catchments may be able to dampen spatial heterogeneities in rainfall (Segond et al., 2007). In the simulations run by Nicotina et al. (2008), the source of rainfall data is from a network of rain gauges. Rainfall resolution is varied by first interpolating the rain gauge data with inverse weighted kriging method to 100m resolution. The 100m resolution data is then upscaled to coarser grid-sizes of 10km and 50km, giving three sets of simulations. Their study uses two catchments of 1560km\(^2\) and 8000km\(^2\) in area. The authors select catchments of relatively large size compared to previous studies. Their choice of larger catchments is based on one their hypotheses being that smaller catchments are closer in size mesoscale rainfall features, and therefore less likely to experience truly heterogeneous spatial rainfall patterns. The results of the Nicotina study show small differences between flood hydrograph peaks, which is more pronounced for the larger (8000km\(^2)\) catchment. A further set of simulations also compares a conservative upscaling of rainfall resolution to a non-conservative upscaling -- i.e. the total volume if rainfall is not necessarily the same post-upscaling. The non-conservative upscaled rainfall resolutions display a greater difference in maximum flood discharge over the three rainfall resolutions than the conservative upscaling method. The authors assert that catchments are more sensitive to the total volume of precipitation than its spatial heterogeneity, although this is perhaps to be expected if the non-conservatively upscaled experiments simply add more water to the catchment at coarser rainfall resolutions. The authors' further experiments with different runoff-generation mechanisms show a much more marked sensitivity in hydrograph response, compared to rainfall spatial heterogeneity. 

From a hydrological perspective, it would appear that getting the total rainfall volume and runoff-generating mechanisms accurately represented in a hydrological model are more important than the spatial pattern of rainfall (Gabellani et al., 2007; Nicotina et al., 2008). However, the approach of previous studies has been to focus primarily on the flood hydrograph during these simulations, which is essentially the water discharge modelled (or measured) at a single point at the catchment outlet. Very few studies, if any, have properly addressed the 2D spatial extent of floodwaters in response to spatially variable rainfall inputs over a catchment. It seems an odd omission to investigate a boundary condition that is by definition spatially heterogeneous over three dimensions (the areal spatial pattern of a rainstorm, as well as the storm depth or intensity), and then to reduce the output to a modelled parameter at a single \(x,y\) coordinate on the model domain. This could be remedied in future research projects.

Intuitively, one might expect that in a river catchment system with its well defined boundaries and singular output point, that any mass-conserving model would produce similar results given water inputs of equal volume (here, I am excluding the non-conserving rainfall upscaling method used by Nicotina at al., 2008). The details of interest may lie in what goes on inside the model domain, rather than what comes out the outlet point. Nevertheless, the work done by the hydrological modelling community has laid some of the foundations for using spatially variable rainfall data in 2D landscape evolution models. A range of data input sources, and interpolation methods that have been successful in hydrological modelling. Some of the basic findings will also help guide the research in the later chapters, and development of an existing landscape evolution model in Chapter 4.

\subsection{Landscape evolution models}

Few of the currently available numerical landscape evolution models explicitly allow the user to vary the spatial distribution of rainfall across the model domain (Valters, 2016). At longer timescales, it can be argued that spatial variation in climatic conditions such as rainfall will eventually be averaged out over centuries and millennia, in effect negating any variation in rainfall spatial patterns (Solyom and Tucker, 2007; Tucker 2010). However, this assumption only holds true if we believe that storm location and rainfall patterns bear no relation to the underlying topography of a landscape or river catchment. In other words, the assumption is that on the short term there is no orographic influence, and on the longer term, that there is no link between evolving topography and evolving weather patterns in a region. Only in recent years, and in a select few studies, have geomorphologists begun to question this assumption. As interest in this question has grown, models have evolved to accommodate this feature. At the short term end of the landscape modelling spectrum (days to centuries), the latest releases of the CAESAR-Lisflood model (Coulthard et al., 2014) now allow for spatially variable rainfall input data. 

\paragraph{Coulthard and Skinner (2016)}

In a sensitivity study that systematically varied the rainfall input data spatial resolution, Coulthard and Skinner (2016) assessed landscape evolution model sensitivity in terms of sediment and water flux, and the spatial distribution of erosion in a mid-sized upland catchment (415km\(^2\)). Rainfall input data was sourced from precipitation radar, and rainfall data resolution is varied at 5km, 10km, 20km resolutions, as well as a `lumped' input where rainfall is averaged spatially across the whole catchment. When the source data is upscaled to finer resolution, the total volume of rainfall is conserved (in contrast to the non-conserving upscaling methods used by Nicotina et al, 2008). The simulations are run with typical rainfall data that is extended over a 30 year period. Compared to the uniform (lumped) precipitation data, increasing the rainfall data grid resolution increases sediment flux from the catchment. In the case of the highest resolution rainfall simulation (5km), sediment flux increases by over 100\% compared to the uniform rainfall case. Coulthard and Skinner's study separates natural spatial variation in rainfall patterns by randomizing the rainfall cell `tiles' from the precipitation radar data, in an attempt to remove any effects from orography in the catchment. In essence, their study is focused solely on the effects of rainfall data resolution alone, rather than the spatial patterns of rainfall in nature, which are often influenced by topography. The rainfall field randomising technique minimise biases from naturally occurring organisation in storm cells and orographic rainfall enhancement. 

\paragraph{Von Ruette, et al (2014)}

So far in this chapter, the discussion has been on landscape evolution models and studies that focus on hydrological, fluvial, and hillslope erosional processes. Numerical models of whole-landscape evolution have a recognized bias towards temperate-humid landscapes (Pazzaglia, 2004; Tucker and Hancock, 2010; Valters, 2016) and tend to focus on a limited gamut of geomorphic processes: hydrology, fluvial erosion, hillslope evolution, and sediment transport. However, the sensitivity of other landscape processes may well be sensitive to the spatial distribution of rainfall over a landscape. Landsliding is an often overlooked, yet important process in landscape evolution and frequently omitted in numerical models (Tucker and Hancock, 2010; Valters, 2016). Von Ruette et al. (2014) investigate the sensitivity of shallow landslide initiation to the spatial distribution of rainfall in a catchment, using a physical based catchment-scale landscape evolution model designed specifically for investigating landslide triggering, the \emph{CHLT} model (von Ruette et al, 2013). In their modelling study, they examine the initiation of shallow landslides under spatially uniform rainfall and a coarse grid-based spatially variable rainfall input, from a real event occurring in 2002. The rainfall input data is a product of integrated rain gauge data and rainfall radar measurements. As the coarseness of the data is high relative to the size of the study catchment, the authors use an inverse distance weighting interpolation method\footnote{An interpolation that gives preferential weighting to points that are closer to each other. The measured values closest to the prediction point of interest have more weighting, which diminishes with distance from the point of prediction.} to downscale the data to a 2.5m grid cell size, the same resolution as the digital elevation model data used in the study. The authors generate a further set of simulations with a set of artificial rainfall input grids at 500m grid cell size. In the model of landslide initiation in the authors' model, the main sensitivity is the rainfall intensity and the infiltration capacity of the soil. If rainfall intensity is too high, water will runoff before it can fully infiltrate the soil; there exists a sweet-spot where rainfall intensities are low enough that the soil will become saturated more readily, and more landslides will be initiated. In the simulations run with equivalent rainfall intensities, spacial heterogeneity exerts some control over the distribution of landslides, as certain grid cells experience high rainfall rates, whereas others experience lower rainfall rates, closer to the rainfall rate `sweet-spot', and consequently more landslide initiation. The findings of the von Ruette (2014) study are complex; sensitivity of landsliding initiation to rainfall spatial heterogeneity is dependent on a number of other conditions such as soil moisture capacity, infiltration rate, rainfall rate, and rainfall intermittency. Rainfall spatial distribution in a catchment exerts a control on whether these conditions will be optimal for landslide intitiation, since it controls local rainfall intensities. Von Ruette et al. conclude that both the spatial distribution of landslides and the total number of landslides triggered are sensitive to the spatial distribution of rainfall in a catchment, assuming other conditions such as infiltration capacity are near-uniform across the catchment.

\subsubsection{Longer term landscape evolution}

\paragraph{Solyom and Tucker (2007)}
Landscape evolution sensitivity to rainfall detail over much longer timescales, on the order of 100kyrs and greater, has been explored to a limited extent by a few studies. Solyom and Tucker (2007) investigate how limited storm size relative to the size and shape of the drainage basin, effects the evolution of landscape topography. In their model, storm cells are represented as circular patterns with peak rainfall intensities at the centre of the circle, decaying exponentially from the centre:

\begin{equation}
I = I_0 \exp(-L_s/L_0)
\end{equation}

where I is he rainfall intensity at a given point in the storm cell, \(I_0\) is the rainfall intensity in the centre of the storm, \(L_s\) is the distance from the centre of the storm to a given point in the storm cell and \(L_0\) is a characteristic length scale associated with the spatial decline of rainfall intensity.

Orographic effects on rainfall enhancement are excluded in the model. In Solyom and Tucker's simulation, a set of idealised diamond-shaped catchments are varied in their elongation (length-width ratio), while being subjected to a steady non-uniform rainfall field described by the exponential decay function, centred at the middle of the diamond-shaped catchment. The exact implementation details in the model code is not revealed by the authors of the study. Their simulations reveal that in general non-uniform rainfall patterns introduce a catchment-shape sensitivity to rainfall-runoff production, which in theory should effect the size and distribution of geomorphic processes throughout the catchment as well. The authors do not present examples of topographies generated by the model, but instead show the total catchment discharge in non-dimensionalised form (\(Q_p/A*I_0\)) compared to non-dimensionalised catchment length (\(L/\sqrt{A}\), where A is the catchment area). Their simulations indicate that the greatest sensitivity occurs when the size of the storm decline rate \(L_0\) is about half of the catchment radius. Solyom and Tucker's interpretation of this is that if storm intensity declines very rapidly over space, i.e. the storm cell is small, then the majority of runoff production occurs in the vicinity of the storm cell, and is therefore insensitive to the shape of the catchment (assuming the storm falls near the centre of the catchment.) If the storm intensity decline rate is small relative to the scale of the catchment then in contrast the catchment is relatively insensitive to catchment shape.

\paragraph{Han and Gasparini (2015)}

A more explicit look at the way topography is influenced by spatial variation in rainfall patterns is found in the recent work of Han and Gasparini (2015). Building on earlier work by Roe et al. (2004), who found the geometry of river long profiles to exhibit sensitivity  to an orographic rainfall feedback mechanism, they explore the sensitivity of the whole landscape over a 2D domain. Modifying the CHILD landscape evolution model (Tucker et al., 2001), they develop a parameterisation scheme for orographic rainfall based on the model of  Smith and Barstad (2004). In their implementation of Smith and Barstad's model, the user controlled variables governing rainfall production are given in Table 3.1. The model offers considerable control over many meteorological variables determining orographic rainfall. In a series of simulations under differing rainfall conditions, the authors find only a slight sensitivity of the concavity of the main trunk channels under spatially variable rainfall. They conclude that channel concavity is not generally sensitive to to orographic rainfall patterns, in contrast to the 1D profile model of Roe et al. (2002) which showed much greater sensitivity. The more revealing topographic metrics were found in planform study -- both the hypsometric integral\footnote{A measure of the fraction of a catchment above a given elevation, describing the distribution of elevations over the catchment. See Brocklehurst and Whipple (2004); Cohen et al. (2008).} and the channel steepness index\footnote{A measure of channel steepness normalised to drainage area; See Wobus et al. (2006).} were found to be more strongly linked to the orographic rainfall gradient. 

\label{HanParameters}
\begin{table}
\begin{tabular}{|c|c|}
\hline 
\textbf{Parameter} & \textbf{Units}  \\ 
\hline 
Initial cloud water column density & kg m \(^{-2}\) \\ 
\hline 
Initial hydrometeor column density & kg m \(^{-2}\) \\ 
\hline 
Time constant for conversion from cloud water to hydrometeors & seconds \\ 
\hline 
Time constant for hydrometeor fallout & seconds \\ 
\hline 
Wind speed & m s \(^{-1}\) \\ 
\hline 
Mountain half width & metres \\ 
\hline 
\end{tabular} 
\caption{User defined parameters in Han and Gasparini's (2015) orographic rainfall model implemented in CHILD.}
\end{table}

In the model domain, rainfall input values for each node are now calculated individually, rather than the uniform rainfall field used in standard versions of CHILD. The calculation is based on a number of factors including the elevation of the current grid node, the direction of the prevailing wind, and factors relating to water content in the atmosphere. As the elevation of grid nodes can change as topography evolves throughout the simulation, and rainfall inputs depend on the elevation of each node, there is an explicit feedback mechanism between orographic precipitation and landscape evolution represented in the model. 

\subsection{Summary of current model capabilities}

The capabilities of landscape evolution models have evolved in tandem with research needs in a piecemeal fashion. As climate change has become an important factor in driving research needs and interests, landscape evolution models have evolved themselves to cater for a range of climatic parameterisations at a range of time scales. Two-dimensional\footnote{Or 2.5-dimensional, if the elevation variable is considered a limited 3rd dimension, in the sense that elevation can go up or down in landscape evolution models, though the underlying process representation remains restricted two-dimensions in the \(x,y\) plane. For example water flow and sediment transport is not fully realised in 3D in any current landscape evolution model.} numerical models are increasingly used for forecasting and predictive purposes, as well as just answering theoretical research driven questions. Despite their potential however, 2D models of landscape evolution are only beginning to be developed to allow detailed spatial variation in many of the climatic variables, such as rainfall. This is seen in the CHILD landscape evolution model work of Han and Gasparini (2015) as well of the development of CAESAR-Lisflood (Coulthard and Skinner, 2016) to simulate spatially variable rainfall input fields. 

Recent advances in landscape evolution modelling have coupled hydrological model components with the core erosional process modules to produce truly hydrodynamic models that do not assume steady state discharge. For example the CAESAR-Lisflood model (Coultard et al., 2013), Landlab modelling framework (Tucker et al., 2015), and tRIBS model [CITE] all contain forms of distributed hydrological models to simulate the transfer of water as well as sediment between grid cells or nodes. At longer timescales, the meteorological processes representing rainfall over a landscape have been parameterised, though the detail of these parameterisation schemes can be quite sophisticated (e.g. Han and Gasparini, 2015).

\section{Research needs for landscape evolution modelling}

The sensitivity of landscape evolutionary processes to the spatial details of climate and precipitation is still relatively unexplored. Though the subject is more advanced in purely hydrological studies (Krajewski et al., 1991; Smith et al. 2005; Segond et al. 2007; Nicotina et al. 2008), there is still a lack of agreement on when sensitivities to rainfall heterogeneities become most pronounced, given the dependence on other aspects of catchment hydrology. The role of runoff generating mechanisms, the influence of vegetation, the influence of groundwater routing pathways, are all affected by the spatial distribution of rainfall in a catchment, yet further investigations into these competing factors are required to reach more consensus among the hydrological community. Though some authors claim there is an insensitivity of hydrological processes to rainfall heterogeneity over a catchment (Krajewski et al. 1991; Smith et al. 2005), a key difference in landscape evolution modelling is that many erosional processes are threshold dependent. When rainfall is uniformly applied across a catchment model, the shear stresses generated by water runoff and river discharge tend to follow a uniform distribution as well. Findings by Coulthard and Skinner (2016) find a pronounced sensitivity to rainfall data resolution in term of sediment flux from a catchment (upto 100\% increases), in contrast to the relatively small differences observed in purely hydrological models (e.g. Nicotina et al. 2008). Apart from the Coulthard and Skinner (2016) paper, no other studies have been found that systematically explore landscape evolution model response to rainfall data resolution. Studies have yet to explore the effect of different spatial patterns of rainfall on the geomorphic impacts of single severe storms. 

With regards to data sources for rainfall input into landscape evolution models, the most typical source is rainfall gauge data, for single sites or sparse networks across a catchment. Rainfall radar has also been explored as a potential source offering higher spatial resolution that most rain gauge data typically available (Coulthard and Skinner, 2016). Other potential sources include output from numerical weather prediction (NWP) models, or the use of artificial weather generators. These two sources offer the potential to explore a variety of different spatial patterns of rainfall data, without having to source them directly from historic events. High resolution rainfall radar data only goes back [XX] number of years [CITE] for example. With methods using NWP models to simulate idealised weather conditions, or using weather generators, researchers have the potential to explore sensitivity to the spatial patterns of rainfall for a variety of meteorological conditions, and the potential to systematically explore different distributions of rainfall on landscape evolution. 

There is still a great deal of unexplored ground for developing landscape evolution models beyond their current capabilities.  Developments are needed to accommodate further types of spatially variable climatic input data and their interpolation (e.g. von Ruette et al, 2014; Coulthard and Skinner, 2016), to develop new feedback models between topography and rainfall generation (e.g. Han and Gasparini, 2015), new parameterisations of storm cell morphology (e.g. Solyom and Tucker, 2007), and to develop models to take advantage of high-performance computing facilities. %(Valters \& Coulthard, 2016/7). 

\textit{More on the research needs here...Perhaps an itemised summary of outstanding questions yet to be answered.}

\subsection{Technological advances}

Landscape evolution modellers have in general been reluctant to take advantage of emerging technology or high performance computing systems to explore bigger problems, or to explore uncertainty in model output through ensemble simulations. By way of contrast, in fields such as meteorology, mineralogy, particle physics, and engineering, the use of high-performance compute facilities is commonplace. In part, this is due to many problems in landscape evolution modelling stemming from a lack of agreement over geomorphic process laws. There is still considerable uncertainty over which geomorphic `laws' are best suited to represent certain natural processes, and the answer can be dependent on the environment being studied. As such, modelling simulations in landscape evolution have often focused on investigating the big-picture, broad-brushed questions about how landscapes evolve as a supplement to empirical field based studies. Geomorphologists, perhaps quite justifiably, have not yet required large-scale computing facilities used in other fields, for their questions can be answered satisfactorily with reduced complexity numerical models. This is especially true as a large body of numerical landscape evolution modelling is used in an exploratory manner (Tucker et al, 2010, some other citations tyo go here...and in this paragraph) -- geomorphologists have been accused by some (Hancock et al., 2003; Pelletier, 2015) of being satisfied simply if their modelled landscape "\textit{looks about right...}" The era of purely qualitative geomorphology has long since passed, and new quantitative methods that can be applied such as the use of topographic metrics should employed. Returning to the comparison with other fields and their use of high-performance computing (HPC), these fields often suffer uncertainty that geomorphology does in the choice of process law or parameterisation used in a numerical simulation. However, this does not stop them from judicious employment of HPC. In fact, one of the strengths of HPC facilities is the capability to assess many hundreds, if not thousands, of scenarios in ensemble simulations -- addressing the uncertainty in  process laws and model parameters that have been noted by others in the modelling field (Tucker and Hancock, 2010; Pelletier, 2015).  A drive towards making use of high performance computing technologies is needed in geomorphology.
