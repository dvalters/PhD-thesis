\chapter{Rainfall representation in current landscape evolution models}
\label{chapter_RainfallInLEMs}
\chaptermark{Rainfall representation in current models}

%\begin{chapquote}{Claude Monet \textit{}}
%``Landscape does not exist in its own right, since its appearance changes at every moment; but the surrounding atmosphere brings it to life'' 
%\end{chapquote}

In the context of numerical landscape evolution models, rainfall input is defined as the quantity of water added to a surface cell or node, or to the whole model domain, over given time period. In practise, no landscape evolution or hydrological models represent rainfall in the sense of it falling from the sky and hitting the ground. The model is only `aware' of rainfall input once it is added as quantity to the hydrological component of the model.  While this may seem a somewhat trivial point, the impact of individual rain drops on the land surface is known to be an important contributor to surface erosion. Rain-splash erosion, as it is termed, is a well-studied phenomenon \citep{morgan1978field, meyer1981rain}. The interaction of raindrops with the surface is complex; it depends on the size of raindrop \citep{morgan1998european}, falling velocity \citep{park1983rainfall}, angle of attack \citep{pedersen1995influence}, soil exposure, soil mineralogy, and cohesion of the soil surface \citep{luk1979effect}. All of these factors could affect both erosion on the landscape hillslopes, as well as the route that water takes before it enters rivers, where fluvial erosion can occur. 

In physical analogue models of landscapes, the generation of rainfall implicitly accounts for some of the above factors in rainsplash erosion, such as drop size, because of the physical need to generate a rainfall source from a position above the surface, such as through a fine-meshed sprinkler \citep[e.g.][]{hancock2001use}. Geomorphologists using physical analogue models of landscape evolution attempt a degree of rainfall realism by ensuring the raindrops they generate are reasonably well scaled to the size of their landscape analogue \citep{meyer1994rainfall}. By contrast, numerical models of landscape evolution begin their representation of rainfall input at the surface; effectively ignoring any effects of rainfall travel through the atmosphere or its impact on the surface. Conceptually, rainfall input in numerical models is the amount of water that would be added at the surface from one or more (usually many more!) raindrops, once they have reached the ground. It ignores any effects from the physical collision raindrops make with the ground. This simple conceptual model of rainfall input is used throughout the rest of this chapter when referring to rainfall input in numerical landscape evolution models.

\section{One-dimensional models}

Isolated aspects of landscape evolution and hydrology can be studied using one dimensional models of features such as hillslopes profiles and longitudinal river profiles, or storm hydrographs in the case of hydrology. Though the work in this thesis focuses on 2D models, it is useful to consider the work done by others investigating the feedback from rainfall variability on 1D models of landscape evolution, before progressing to full 2- or 2.5D\footnote{Occasionally, the terms 2D and 2.5D are used interchangeably when referring to landscape evolution models, although in effect they both produce what looks like a `3D' terrain surface from their output. The `third' dimension (or extra 0.5D in 2.5D terminology) comes from the fact that the elevation variable can be used to reconstruct a 3D picture of the landscape based on the value for each grid cell or node. In practice, nearly all of the process models in landscape evolution models are 2D, e.g. water routing over the surface does not account for turbulent flow in \textit{x}, \textit{y} and \textit{z} directions, such as in computational fluid dynamic models. Sediment transport does not account directly for 3D particle motion or collisions between particles.} models over an \(x,y\) model domain. 


%REWORD FOCUS
One dimensional models simplify aspects of the landscape system, isolating key components such as the longitudinal profile of a river or hillslope.\citet{Roe2002} modify a simple 1D model for river profile evolution \citep{seidl1992problem,howard1994detachment} to incorporate a feedback for orographic precipitation based on increasing elevation along a steepening river profile. Their precipitation feedback model accounts for two precipitation regimes: the first typical of midlatitude, shallower, and narrower mountain ranges such as the West coast of North America, and one for broader and taller ranges such as the Sierra Nevada, European Alps or the Southern Alps of New Zealand. The former represents rainfall patterns that are dominated by the prevailing upslope winds, increasing precipitation with distance upstream, whereas the latter represents environments where atmospheric moisture content exerts more control over precipitation, resulting in decreasing rainfall at higher elevations, and a rainfall shadow on the leeward side of the range. In a later work \citep{Roe2003} the 1D model incorporating orographic rainfall feedback is extended to the 1D relief structure over a transect of a mountain range. The maximum relief is found to be strongly dependent on the type of precipitation regime chosen; the prevailing upslope wind regime favouring lower relief, symmetric mountain ranges, and the atmospheric moisture-limited regime favouring higher relief mountain ranges.

Further 1D models have been developed to determine the relative importance of rainfall variability compared to other boundary conditions, such as tectonic uplift or base level fall. The 1D river profile model of \citet{Wobus2010} uses a transport-limited formulation of river profile evolution \citep{meyer1948formulas} with a simple parameterisation of rainfall based on modifying the exponent to the discharge-area approximation given by:

\begin{equation}
q_w = k_qA^c
\end{equation}

\noindent
where \(q_w\) is the water discharge, \(k_q\) a dimensional coefficient, \(A\) the contributing drainage area, and \(c\) the exponent that relates which portions of the drainage basin contribute to gathering precipitation and converting it to water discharge. A decrease in \(c\) represents a shift to more rainfall being gathered in the upper reaches of the stream. An increase in \(c\) represents rainfall being gathered in the lower reaches. The situation where \(c = 1\) implies rainfall input is equal along all sections of the river profile. The end result of this formulation is perhaps intuitive: more rainfall input in the upper reaches of the stream (decrease in \(c\) ) results in more incision in the headwaters. However, the study reveals a key difference in the way that climatic and tectonic signals propagate along a river channel. Numerical results show that rainfall-driven perturbations propagate from the channel head downstream, whereas tectonic perturbations invariably propagate from base-level upwards towards the channel head. The authors, however, reach this conclusion without simulating the scenario where there is more contributing rainfall from the lower reaches, i.e. the value of \(c\) is higher. However, given the setting of the study (streams draining a mountain front) it is perhaps reasonable to assume an increasing precipitation rate as elevation increases with distance into the mountain range.

River channel profiles are not the only markers of landscape evolution, though they do dominate the range of 1D modelling studies investigating sensitivity to the spatial distribution of rainfall \citep{Tucker2010}. Hillsope erosion rates are known to be sensitive to the spatially averaged rainfall rate, especially where vegetation cover is low. \citep{Owen2011}. Hillslope bedrock erosion decreases according to a power law as mean rainfall rates decrease, from semi-arid to hyperarid environments. In general though, the study of hillsope sensitivity to the spatial distribution of rainfall remains under-studied, particularly in the case of 1D profile evolution.

In the one-dimensional modelling studies discussed, there is a key limitation, which is often acknowledged by the authors. Channels profiles in 1D form are modelled without their tributary streams. The main stem of the channel is assumed to be representative of the entire catchment as a whole. This implies that tributary channels, and hillslopes feeding the main channel, experience the same precipitation patterns, or that differences between the main channel and its contributing water sources can be ignored. Rainfall spatial variability from tributary channels, or from runoff over hillslopes is lost, or `smeared-out' \citep{Roe2002}. The effects of water routing within a drainage network are also lost, and interesting relationships between rainfall distribution, river network connectivity and erosion are potentially overlooked. Complex parameterisations of rainfall production are often reduced to a single number or exponent in an equation describing the evolution of the landform profile of interest. Rainfall spatial patterns are often complex over correspondingly complex terrain, and only 2D models may suffice to fully explore the sensitivity of landscape process and form to rainfall spatial distribution.


%\subsection{2D models}
%Early 2D numerical models of landscape evolution were often driven by single process laws of fluvial incision, and the topography that resulted from them was a product of the parameters in the fluvial incision laws. Simple fluvial incision laws, implemented in 2D numerical models resulted in topography broadly similar to the fractal patterns of river networks observed in nature [CITE Ahnert/Turcotte], with the hillslope features between neighbouring river channels been formed by what was `left behind' from fluvial incision patterns. In other words, separate process laws were not implemented to describe the typically diffusive processes observed in hillslope formation. [Roerring, Hurst, citations from Hurst]
%
%A typical form of the simple stream power law for fluvial incision takes the form:
%
%\begin{equation}
%E = KA^mS^n
%\end{equation}
%
%\noindent
%where \(K\) is termed the coefficient of erodibility, and is a catch-all term for climatic processes (amongst others) including the role of rainfall on the fluvial incision process. The \(K\) term itself can be considered a proxy for rainfall variation over time, assuming all other factors remained constant\citep[e.g.][]{rinaldo1995geomorphological}. However in this approach, the connection between physical processes is lost, or abstracted from 
%
%Another simple model is the excess shear stress model for fluvial incision, where the incision or erosion rate, \(E\) is given as a function of shear stress, \(\tau\) above a threshold level, \(\tau_c\):
%
%\begin{equation}
%E = k_e(\tau^a -  {\tau^a}_c)
%\end{equation}
%
%\noindent
%With this simple model of landscape evolution, one of the first studies to study the 2D evolution of topography under varying climatic conditions was that of Rinaldo (1993). The study implemented a cyclic variation through time on the parameter of critical shear stress, the threshold for erosion, \(\tau_c\). Since shear stresses driving incision are determined by river discharge, which in turn is controlled by rainfall input, the cyclical variation in critical shear stress, \(\tau_c\) can be used a proxy for temporal variations in rainfall over the catchment at geological timescales. When the value of \(\tau_c\) is low during the model this effectively represents a period of high rainfall intensity, and when \(\tau_c\) is high this represents a period of lower intensity rainfall (Rinaldo, 1993). In the resulting topographies from these simulations, drainage density and fractal dimension were shown to increase in response to a decrease in critical shear stress, or an increase in rainfall input over time, assuming other factors such as uplift remain constant.

%Other studies to expand on:
%\begin{itemize}
%\item CHILD (Tucker) - Precipitation Stochastic Model.
%\item Colberg and Anders (2003)
%\item Solyom and Tucker (2004) - is this distributed or not? Probably not because rainfall-runoff is a parameterisation.
%\item Solyom and Tucker (2007)
%\end{itemize}

\section{Two-dimensional models}
%Distributed models\footnote{I borrow the term `distributed model' here from hydrological modellers.}
Two-dimensional models are grid-cell based (or based on a grid of connected `nodes') and allow certain variables to vary spatially across the model domain, from cell-to-cell or node-to-node. There are comparatively few landscape evolution models that allow spatially variable rainfall input to be distributed across the model domain, and some of the examples discussed here are from purely hydrological models. However, the principle of modelling spatially-distributed rainfall remains the same and there are potential applications in hydrological modelling that can be extended to landscape evolution modelling purposes. 

\subsection{Hydrological models}
% Define distributed models here?
In the world of hydrological modelling, distributed\footnote{\textit{Distributed} hydrological models take into account the spatial distribution of meteorological inputs like rainfall, and other distributed variables such as soil moisture, vegetation, and land use. `Distributed' is a term usually applied to hydrological models and less so to landscape evolution models, though in practice both can be said to be types of distributed model if inputs are spatially variable across the model domain.} rainfall inputs are more commonplace. A range of meteorological input data sources have been used to drive distributed hydrological models. Three main sources of spatial rainfall data commonly used are dense-network rainfall-gauge data, precipitation radar, and precipitation outputs from numerical weather prediction models. Each one of these sources has a range of merits and demerits as a raw data source (further discussed in Chapter \ref{chapter_metdata}), but the discussion here focuses on their suitability as spatially heterogeneous rainfall datasets for numerical landscape evolution models, rather than an appraisal of their relative merits in reporting precipitation distribution.

Precipitation data generated by numerical weather prediction models has been successfully used in distributed hydrological models to make hydrological forecasts, as well as to analyse historic flooding events. \citet{hay2006one} use the MM5 model (Mesoscale Meteorological model)\footnote{A precursor to the Weather Research and Forecasting model, WRF.} to generate gridded rainfall data over a five-year period. The rainfall data is used to drive the PRMS distributed hydrological model (the Precipitation Runoff Modelling System) over a corresponding five-year period. The numerical weather prediction model is run at grid cell spacings of 20km, 5km, and 1.7km, the finest of which resolves individual valleys and massifs, and captures the resulting rainfall patterns over the catchment at high resolution. The study also compares the way that rainfall input zones in the hydrological model are represented. In the hydrological model, different zones of rainfall input can be defined along natural topographic boundaries, which are termed \textit{Hydrological Response Units}. These rainfall zone units tend to follow sub-catchment boundaries within the main catchment watershed. Alternatively, the catchment can be divided up more simply into rainfall input zones corresponding to a regularly spaced grid at a cell-spacing that matches the resolution of the input data.
In general, increasing rainfall input resolution in the \citet{hay2006one} study results in a greater accuracy when compared with observed river discharge values. Using irregular-shaped hydrological response units based on natural sub-catchments, rather than a regular gridding of input data, results in better agreement with observation. However, as resolution increases towards the 1.7km grid-cell spacing, the difference seen from using irregular shaped hydrological response units and regular grids of comparable resolution decreases. 

A study that uses high resolution numerical weather prediction (NWP) model data to drive a hydrological model \citep{younger2008usability} tests the suitability of rainfall forescast data for making hydrological predictions and improving flood forecasting. High resolution (250m grid spacing) simulations using the United Kingdom Met Office Unified Model are used to generate input rainfall data to drive a TOPMODEL-based hydrological model \citep{beven2001dynamic}. The semi-distributed hydrological model \textit{Dynamic-TOPMODEL} of \citet{beven2001dynamic} groups topographically similar regions of the catchment and calculates runoff-predction for each of the these self-similar zones. The amount of runoff calculated is then assigned to each node in that particular zone. Computationally, this is more efficient than performing runoff calculations for every single grid cell in the catchment domain.
The \citet{younger2008usability} study considers two events, a summer convective rainfall-event and a winter stratiform rainfall event. Although the hydrological simulation using the dense-network of rainfall gauge data produced outputs more closely matched to discharge observations, simulations with the NWP rainfall forecast also produce accurate results. \citet{younger2008usability} highlight the potential of using high-resolution rainfall forecast data to improve flood-forecasting in the future, giving greater prediction lead-in times compared to nowcasting from rainfall radar or real-time raingauge measurements. 

Rainfall data from numerical weather prediction models lends itself well to use as input data for hydrological modelling; it is typically written in a gridded data output format, and if the user has control over both the generation of the NWP output as well as the hydrological or landscape evolution model, generating compatible data formats can be more straightforward.

A consensus has yet to emerge on whether distributed hydrological models are sensitive to the spatial distribution of rainfall input. Even when comparing catchments of similar sizes and environments, many studies are in disagreement \citep{nicotina2008impact}. In terms of the peak discharge and the time to the peak from the onset of heavy rainfall during a flood, the use of spatially heterogeneous rainfall input data appears to have little impact on the predicted hydrographs \citep{krajewski1991monte,shah1996amodelling,shah1996bmodelling}. Antecedent conditions may determine some of the relative sensitivity in catchment hydrological response \citep{shah1996bmodelling}, but only when initial water saturation levels are low. Variability in runoff production mechanisms are thought to be an important control on runoff response \citep{shah1996bmodelling,segond2007simulation}. Whether variability in rainfall heterogeneity also contributes to the runoff response depends on antecedent conditions, as catchments may be able to dampen spatial heterogeneities in rainfall \citep{segond2007simulation}. In the simulations run by \citet{nicotina2008impact}, the source of rainfall data is from a network of rain gauges. Rainfall resolution is varied by first interpolating the rain gauge data with a kriging\footnote{Using a weighted average of neighbouring samples to estimate the unknown rainfall value at a given grid cell location.} method to 100m resolution. The 100m resolution data is then upscaled to coarser resolutions of 10km and 50km, giving three sets of simulations, one for each data resolution. Their study uses two catchments of 1560km\(^2\) and 8000km\(^2\) in area. \citet{nicotina2008impact} select catchments of relatively large size compared to previous studies. The choice of larger catchments is based on one of their hypotheses being that smaller catchments are closer in size to mesoscale rainfall features, and therefore less likely to experience truly heterogeneous spatial rainfall patterns. The results of the \citet{nicotina2008impact} study show small differences between flood hydrograph peaks, which are more pronounced for the larger (8000km\(^2)\) catchment. A further set of simulations also compares a conservative upscaling of rainfall resolution to a non-conservative upscaling, i.e. the total volume of rainfall is not necessarily the same post-upscaling. The non-conservative upscaled rainfall resolutions display a greater difference in maximum flood discharge over the three rainfall resolutions than the conservative upscaling method. \citet{nicotina2008impact} assert that catchments are more sensitive to the total volume of precipitation than its spatial heterogeneity, although this is perhaps to be expected if the non-conservatively upscaled experiments simply add more water to the catchment at coarser rainfall resolutions.  Further experiments in the \citet{nicotina2008impact} study with different runoff-generation mechanisms show a much more marked sensitivity in hydrograph response, compared to rainfall spatial heterogeneity. 

From a hydrological perspective, accurate representation of the total rainfall volume and runoff-generating mechanisms in a hydrological model are more important than the spatial pattern of rainfall distribution \citep{gabellani2007propagation,nicotina2008impact}. However, the approach of previous studies has been to focus primarily on the flood hydrograph during these simulations, which is essentially the water discharge modelled (or measured) at a single point at the catchment outlet. Few studies have properly addressed the sensitivity of the areal extent of floodwaters in response to spatially variable rainfall inputs to a catchment. To aid understanding and prediction of the risk posed to communities by flash flooding, knowing to what degree flood waters are sensitive to spatial variation in rainfall inputs would be a useful, in addition to just predicting the peak flows for a given point in the catchment.

Intuitively, one might expect that in a river catchment system with its well defined boundaries and singular output point, that any mass-conserving model would produce similar results given water inputs of equal volume\footnote{Excluding the non-conserving rainfall upscaling method used by \citet{nicotina2008impact}}. The details of interest may lie in what goes on inside the model domain, rather than what comes out the catchment outlet point. In other words the spatial distribution of floodwaters should be just as important as the water discharge at the outlet. Nevertheless, the work done by the hydrological modelling community has laid some of the foundations for using spatially variable rainfall data in two-dimensional landscape evolution models. A range of data input sources and interpolation methods that have been successfully tested in hydrological modelling and should be transferable to models that also simulate the erosional processes within catchments. Some of the basic findings will also help guide the research in the later chapters, and in extending the development of an existing landscape evolution model in Chapter \ref{chapter_HAIL-CAESAR}.

\subsection{Landscape evolution models}

Few of the currently available numerical landscape evolution models explicitly allow the user to vary the spatial distribution of rainfall across the model domain. At longer timescales, spatial variation in climatic conditions such as rainfall will eventually be averaged out over centuries and millennia, in effect negating any variation in rainfall spatial patterns \citep{solyom2007importance,Tucker2010}. However, this assumption only holds true if we believe that storm location and rainfall patterns bear no relation to the underlying topography of a landscape or river catchment. In other words, the assumption is that on the short term there is no orographic influence, and on the longer term, that there is no link between evolving topography and evolving weather patterns in a region. Only in recent years, and in a select few studies, have geomorphologists begun to question this assumption. As interest in this question has grown, models have been developed to accommodate this functionality. At the short term end of the landscape modelling spectrum (days to centuries), the latest releases of the CAESAR-Lisflood model \citep{coulthard2017caesarlisflood} now allow for spatially variable rainfall input data.
 
Landscape evolution models have been used to show the erosional processes are sensitive to spatial variation in rainfall inputs over timescales of decades to centuries. In a sensitivity study that systematically varied the rainfall input data spatial resolution to the CAESAR-Lisflood model, \citet{coulthard2016sensitivity} assessed landscape evolution model sensitivity in terms of sediment and water flux, and the spatial distribution of erosion in a mid-sized upland catchment (415km\(^2\)). Rainfall input data was sourced from precipitation radar, and rainfall data resolution is varied at 5km, 10km, 20km resolutions, as well as a `lumped' input where rainfall is averaged spatially across the whole catchment. When the source data is upscaled to finer resolution, the total volume of rainfall is conserved (in contrast to the non-conserving upscaling methods used by \citet{nicotina2008impact}). The simulations are run with 10-years of continuous rainfall data from rainfall radar records, looped over a 30-year period. Compared to the uniform (lumped) precipitation data, increasing the rainfall data grid resolution increases sediment flux from the catchment. In the case of the highest resolution rainfall simulation (5km), sediment flux increases by over 100\% compared to the uniform rainfall case. Natural spatial variation in rainfall patterns is removed by randomising the rainfall cell `tiles' from the precipitation radar data, in an attempt to remove any rainfall patterns from orography in the catchment. In essence, \citet{coulthard2016sensitivity} focus solely on the effects of rainfall data resolution alone, rather than the spatial patterns of rainfall in nature, which are often influenced by topography. The rainfall field randomising technique minimises biases from naturally occurring organisation in storm cells and orographic rainfall enhancement. 

Fluvial processes and flooding are not the only geomorphic processes potentially sensitive to the spatial distribution of rainfall. Numerical models of whole-landscape evolution have a recognized bias towards temperate-humid landscapes \citep{Pazzaglia2003,Tucker2010} and tend to focus on a limited range of geomorphic processes: hydrology, fluvial erosion, hillslope evolution, and sediment transport. Landsliding is an often overlooked, yet important process in landscape evolution though it frequently goes unrepresented in numerical models of whole-landscape evolution\citep{Tucker2010}. Landslide triggering is generally addressed in more domain-specfic models, such as \citet{von2014effects}, who investigate the sensitivity of shallow landslide initiation to the spatial distribution of rainfall in a catchment. A catchment-scale landscape evolution model designed specifically for investigating landslide triggering, the \emph{CHLT} model \citep{vonruette2013rainfall} to investigate the initiation of shallow landslides under spatially uniform rainfall and a coarse grid-based spatially variable rainfall input, from an storm event occurring in 2002. The rainfall input data is a product of integrated rain gauge data and rainfall radar measurements. As the coarseness of the data is high relative to the size of the study catchment, an inverse distance weighting interpolation method\footnote{An interpolation that gives preferential weighting to points that are closer to each other. The measured values closest to the prediction point of interest have more weighting, which diminishes with distance from the point of prediction.} is used to downscale the data to a 2.5m grid cell size, the same resolution as the digital elevation model data used in the study. A further set of simulations is carried out with a set of artificial rainfall input grids at 500m grid cell size. The simulations reveal the main sensitivity to landslide initiation is the rainfall intensity and the infiltration capacity of the soil. If rainfall intensity is too high, water will run off before it can fully infiltrate the soil; there exists a `sweet-spot' where rainfall intensities are low enough that the soil will become saturated more readily, and more landslides will be initiated. In the simulations run with equivalent rainfall intensities, spacial heterogeneity exerts some control over the distribution of landslides, as certain grid cells experience high rainfall rates, whereas others experience lower rainfall rates, closer to the rainfall rate `sweet-spot', and consequently more landslide initiation. The findings of \cite{vonruette2013rainfall} show that sensitivity of landsliding initiation to rainfall spatial heterogeneity is dependent on a number of other conditions such as soil moisture capacity, infiltration rate, rainfall rate, and rainfall intermittency. Rainfall spatial distribution in a catchment exerts a control on whether these conditions will be optimal for landslide initiation, since rainfall distribution controls local water inputs at the surface. It is concluded that both the spatial distribution of landslides and the total number of landslides triggered are sensitive to the spatial distribution of rainfall in a catchment, assuming other conditions such as infiltration capacity are near-uniform across the catchment.

\subsubsection{Longer term landscape evolution}

Landscape evolution sensitivity to rainfall detail over much longer timescales, on the order of 100 000 years and greater, has been explored to a limited extent in a few studies. The ratio of storm cell size to catchment shape and size is one factor believed to control the long term evolution of catchment topography and drainage network morphology \citep{solyom2007importance}. In the \citet{solyom2007importance} model, storm cells are represented as perfectly circular features with peak rainfall intensities at the centre of the circle, decaying exponentially from the centre:

\begin{equation}
I = I_0 \exp(-L_s/L_0)
\end{equation}

\noindent
where \(I\) is the rainfall intensity at a given point in the storm cell, \(I_0\) is the rainfall intensity in the centre of the storm, \(L_s\) is the distance from the centre of the storm to a given point in the storm cell and \(L_0\) is a characteristic length scale associated with the spatial decline of rainfall intensity.

Orographic effects on rainfall enhancement are not represented in the model. In Solyom and Tucker's simulation, a set of idealised diamond-shaped catchments are varied in their elongation (length--width ratio), while being subjected to a steady non-uniform rainfall field described by the exponential decay function, centred at the middle of the diamond-shaped catchment. The exact implementation details in the model code is not revealed by the authors of the study. In general, non-uniform rainfall patterns introduce a catchment-shape sensitivity to rainfall-runoff production, which in theory should affect the magnitude and distribution of erosional processes throughout the catchment as well. Examples of topographies generated by the model are not presented in the study, but shown instead is the total catchment discharge in non-dimensionalised form (\(Q_p/A*I_0\)) compared to non-dimensionalised catchment length (\(L/\sqrt{A}\), where \(A\) is the catchment area). The simulations indicate that the greatest sensitivity occurs when the size of the storm decline rate \(L_0\) is about half of the catchment radius. Solyom and Tucker's interpretation of this is that if storm intensity declines very rapidly over space, i.e. the storm cell is small, then the majority of runoff production occurs in the vicinity of the storm cell, and is therefore insensitive to the shape of the catchment (assuming the storm falls near the centre of the catchment.) If the storm intensity decline rate is small relative to the scale of the catchment then in contrast the catchment is relatively insensitive to catchment shape.


A more detailed look at the way topography is influenced by spatial variation in rainfall patterns -- including the effect of orographic rainfall gradients -- is found in the recent work of \cite{han2015measuring}. Building on earlier work by \citet{Roe2002}, who found the geometry of river long profiles to exhibit sensitivity  to an orographic rainfall feedback mechanism, they explore the sensitivity of the whole landscape over a 2D domain. Modifying the CHILD landscape evolution model \citep{Tucker2001}, they develop a parameterisation scheme for orographic rainfall based on the model of \citet{smith2004linear}. In their implementation of Smith and Barstad's model, the user-controlled variables governing rainfall production are given in Table \ref{HanParameters}. The model offers considerable control over many meteorological variables determining orographic rainfall. In a series of simulations under differing rainfall conditions, the authors find only a slight sensitivity of the concavity of the main trunk channels under spatially variable rainfall. They conclude that channel concavity is not generally sensitive to to orographic rainfall patterns, in contrast to the 1D profile model of \citet{Roe2002} which showed much greater sensitivity. The more revealing topographic metrics were found in planform study; both the hypsometric integral\footnote{A measure of the fraction of a catchment above a given elevation, describing the distribution of elevations over the catchment. See \citet{brocklehurst2004hypsometry,cohen2008methodology}.} and the channel steepness index\footnote{A measure of channel steepness normalised to drainage area \citep{wobus2006tectonics}} were found to be more strongly linked to the orographic rainfall gradient. 

\label{HanParameters}
\begin{table}
\begin{tabular}{|c|c|}
\hline 
\textbf{Parameter} & \textbf{Units}  \\ 
\hline 
Initial cloud water column density & kg m \(^{-2}\) \\ 
\hline 
Initial hydrometeor column density & kg m \(^{-2}\) \\ 
\hline 
Time constant for conversion from cloud water to hydrometeors & seconds \\ 
\hline 
Time constant for hydrometeor fallout & seconds \\ 
\hline 
Wind speed & m s \(^{-1}\) \\ 
\hline 
Mountain half width & metres \\ 
\hline 
\end{tabular} 
\caption{User defined parameters in the \citet{han2015measuring} orographic rainfall model implemented in CHILD.}
\end{table}

In the model domain, rainfall input values for each node are now calculated individually, rather than the uniform rainfall field used in standard versions of CHILD. The calculation is based on a number of factors including the elevation of the current grid node, the direction of the prevailing wind, and factors relating to water content in the atmosphere. As the elevation of grid nodes can change as topography evolves throughout the simulation, and rainfall inputs depend on the elevation of each node, there is an explicit feedback mechanism between orographic precipitation and landscape evolution represented in the model. 

\subsection{Summary of current model capabilities}

The technical capabilities of landscape evolution models have evolved in tandem with research needs in a piecemeal fashion. As climate change has become an important factor in driving research needs and interests, landscape evolution models have evolved themselves to cater for a range of climatic parameterisations at a range of time scales. Two-dimensional\footnote{Or 2.5-dimensional, if the elevation variable is considered a limited 3rd dimension, in the sense that elevation can go up or down in landscape evolution models, though the underlying process representation remains restricted two-dimensions in the \(x,y\) plane. For example water flow and sediment transport is not fully realised in 3D in any current landscape evolution model.} numerical models are increasingly used for forecasting and predictive purposes, as well as just answering theoretical research-driven questions. Despite their potential however, 2D models of landscape evolution are only beginning to be developed to allow detailed spatial variation in many of the climatic variables, such as rainfall. This is seen in the CHILD landscape evolution model work of \citet{han2015measuring} as well of the development of CAESAR-Lisflood \citep{coulthard2016sensitivity,coulthard2017caesarlisflood} to simulate spatially variable rainfall input fields. 

Recent advances in landscape evolution modelling have coupled hydrological model components with the core erosional process modules to produce truly hydrodynamic models that do not assume steady state discharge and runoff in a catchment. For example, the CAESAR-Lisflood model \citep{Coulthard2013}, Landlab modelling framework \citep{hobley2017creative}, and tRIBS model \citep{vivoni2011real} all contain forms of distributed hydrological models to simulate the transfer of water as well as sediment between grid cells or nodes. At longer timescales, the meteorological processes representing rainfall over a landscape have been parameterised, though the detail of these parameterisation schemes can be quite sophisticated \citep[e.g.][]{han2015measuring}, incorporating established models of rainfall production and enhancement in mountainous regions \citep{Roe2002,Roe2003,smith2004linear}.

\section{Research needs for hydro-geomorphology}

The sensitivity of hydro-geomorphic processes to the spatial details of climate and precipitation is still relatively unexplored. Hydro-geomorphology is key to understanding long-term landscape evolution as well as the shorter-term risks faced by communities in proximity to rivers, floodplains, and landslide-prone areas. Though the field is more advanced in purely hydrological studies \citep{krajewski1991monte,smith2005field,segond2007simulation,nicotina2008impact} there is still a lack of agreement on when sensitivities to rainfall heterogeneities become most pronounced, given the dependence on other aspects of catchment hydrology. The role of runoff generating mechanisms, the influence of vegetation, the influence of groundwater routing pathways, are all affected by the spatial distribution of rainfall in a catchment, yet further investigations into these competing factors are required to reach more consensus among the hydrological community. Although some hydrological studies there is an insensitivity of hydrological processes to rainfall heterogeneity over a catchment \citep{krajewski1991monte,smith2005field}, a key difference in landscape erosion evolution modelling is that many erosional processes are threshold dependent \citep{snyder2003importance}. Findings by \citet{coulthard2016sensitivity} find a pronounced sensitivity to rainfall data resolution in term of sediment flux from a catchment (up to a 100\% increase), in contrast to the relatively small differences observed in purely hydrological models \citep[e.g.][]{nicotina2008impact}. Apart from the \citet{coulthard2016sensitivity}, no other studies have been found that systematically explore landscape evolution model response to rainfall data resolution. Studies have yet to explore the effect of different spatial patterns of rainfall on the hydro-geomorphic impacts of single severe storms. 

With regards to data sources for rainfall input into landscape evolution models, the most typical source is rainfall gauge data, for single sites or sparse networks across a catchment. Rainfall radar has also been explored as a potential source offering higher spatial resolution that most rain gauge data typically available (Coulthard and Skinner, 2016). Other potential sources include output from numerical weather prediction (NWP) models, or the use of artificial weather generators \citep[e.g.][]{Peleg2014}. These two sources offer the potential to explore a variety of different spatial patterns of rainfall data, without having to source them directly from historic events. With approaches using NWP models to simulate idealised weather conditions, or using weather generators, or data from rainfall radar archives, researchers have the potential to explore sensitivity to the spatial patterns of rainfall for a variety of meteorological conditions, and the potential to systematically explore different distributions of rainfall on landscape evolution. 

There is still a great deal of unexplored ground for developing landscape evolution models beyond their current capabilities.  Developments are needed to accommodate further types of spatially variable climatic input data and their interpolation to different scales \citep[e.g.][]{von2014effects,coulthard2016sensitivity}, to develop new feedback models between topography and rainfall generation \citep[e.g.][]{han2015measuring}, new parameterisations of storm cell morphology \citep[e.g.][]{solyom2007importance}, and to develop models to take advantage of high-performance computing facilities.

\subsection{Technological advances and needs}

Landscape evolution modellers have in general been reluctant to take advantage of emerging technology or high-performance computing (HPC) systems to explore bigger problems, or to explore uncertainty in model output through ensemble simulations. By way of contrast, in fields such as meteorology, mineralogy, particle physics, and engineering, the use of high-performance compute facilities is commonplace. In part, this is due to many problems in landscape evolution modelling stemming from a lack of agreement over geomorphic process laws. There is still considerable uncertainty over which geomorphic `laws' are best suited to represent certain natural processes, and the answer can be dependent on the environment being studied. As such, modelling simulations in landscape evolution have often focused on investigating the big-picture, broad-brushed questions about how landscapes evolve as a supplement to empirical field based studies. Geomorphologists, perhaps quite justifiably, have not yet required large-scale computing facilities used in other fields, as their questions can be answered satisfactorily with reduced complexity, and less computationally demanding numerical models. This is especially true as a large body of numerical landscape evolution modelling is used in an exploratory manner \citep{hancock2003effect,lancaster2003you,Tucker2010}, or to make observations that are partly qualitative. However, geomorphology is moving away from a purely qualitative era as a range of quantitative methods have been developed and deployed in recent studies incorporating landscape evolution modelling \citep[e.g.][]{Attal2011a,mudd2017detection}, such as absolute dating methods of landforms and sediments, as well as digital topographic analysis. Returning to the comparison with other fields and their use of high-performance computing, these fields often suffer uncertainty that geomorphology does in the choice of process law or parameterisation used in a numerical simulation. However, this has not prevented them from judicious use of HPC in carrying out numerical simulations. In fact, one of the strengths of HPC facilities is the capability to assess many hundreds, if not thousands, of scenarios in ensemble simulations. This use of HPC has helped to quantify the uncertainty in process laws and model parameters, a problem which geomorphology is not immune to \citep{Tucker2010,pelletier2015forecasting}. A drive towards making use of high-performance computing facilities is needed in geomorphology as problem sizes grow through the continued expansion in data coverage and resolution.
