%\title{Numerical modelling of catchment hydrogeomorphological sensitivity to rainfall spatial distribution and erosion law parameterisation during convective rainfall events in the UK}

\title{Numerical modelling of hydrogeomorphological sensitivity to rainfall and erosion law parameterisation during convective rainfall events in the UK}

\author{Declan A. Valters}
% Faculty of Life Sciences people should comment the next line out
\school{Earth and Environmental Sciences}
\faculty{Science and Engineering}
\def\wordcount{74333}

% Uncomment the line below to suppress the `List of Tables' page (optional)
%\tablespagefalse

% Uncomment the line below to suppress the `List of Figures' page (optional)
%\figurespagefalse

% Uncomment the line below to use a customised Declaration statement
%\def\declaration{All the work in this thesis has been sourced from Google.}

\beforeabstract
The contribution of this thesis is twofold: 1) the development of a hydrodynamic landscape evolution model for use on high-performance computing systems and 2) assessing the sensitivity of hydrogeomorphic processes to high-resolution rainfall input data using the model.

The thesis addresses a limitation in numerical landscape evolution models regarding how spatial variation in rainfall is represented or parameterised within such models. Typically, landscape evolution models forsake a realistic representation of rainfall patterns in favour of a simpler treatment of rainfall as being spatially homogeneous across the model domain. This simplification of rainfall spatial variability is still made despite the fact that many geomorphological processes are sensitive to thresholds of sediment entrainment and transport, driven by the distribution and movement of water within the landscape. 

The thesis starts by exploring current limitations in rainfall representation in landscape evolution models, and assesses various precipitation data sources that could be potentially used as more realistic rainfall inputs to landscape evolution models. A numerical model of landscape evolution is developed for deployment on high-performance parallel computing systems, based on the established CAESAR-Lisflood model \citep{Coulthard2013}. The new model code is benchmarked, showing performance benefits compared with the original CAESAR-Lisflood model it is based on.

The model is applied to assessing the sensitivity of flood-inundation predictions, sediment flux, and erosion distribution within river catchments to spatial variation in rainfall during extreme storm events. Two real storm events that caused localised flash flooding in the UK are used as test cases: the Boscastle storm of 2004 and the North York Moors storm of 2005. 

Flood extent predictions and river discharges are found to be sensitive to the use of spatially variable input rainfall data, with high-resolution rainfall data leading to larger peak flood discharges. However, the differences are less pronounced in smaller catchments. The role of sediment erosion during large floods is also assessed, but it is found to play a minor role relative to spatially variable rainfall data. In contrast, the geomorphological response of catchments to single storm events is shown to be less sensitive to the spatial heterogeneity of rainfall input and controlled more strongly by the choice of erosional process parameterisation within the model. Nonetheless, spatial variability in rainfall data is shown to increase sediment yields during flash flood simulations.

%Finally, a software framework is presented for driving a landscape evolution model with input from a numerical weather prediction model. Using high-resolution NWP simulations of 250m grid spacing, mesoscale rainfall features are resolved at the river catchment scale. The UK storm case studies described in the previous chapter are re-assessed using data from NWP simulations of each case to investigate the response of catchments to the mesoscale details of severe rainfall events.
%To explore the effects of rainfall spatial variability at longer timescales of topographic evolution, the CHILD landscape evolution model (Tucker et al., 2001) is developed to incorporate a model of spatially-limited storm morphology. The rainfall model is designed to represent the typical intense rainfall associated with convective storm cells.


%
%
%From a meteorological perspective, precipitation patterns are observably non-uniform at a variety of scales, due to effects such as the orographic enhancement of rainfall, as well as the spatially limited nature of convective storm cells that bring intense rainfall. In temperate regions, many geomorphic processes are ultimately driven by the movement of water in the landscape, such as fluvial incision and sediment transport.
%
%Write your abstract here: Remember, it must fit on this A4 page and should
%describe contents of the thesis/dissertation. Here might also be a good place
%to indicate what you have achieved in the thesis/dissertation and, in the
%case of a PhD, what new results you have discovered. Note that for a PhD
%single-spacing is used throughout the Abstract, including displayed equations
%\[
%e = mc^{2}
%\]
%as for the above example.

\afterabstract
%
%% The next part is optional; however it is a good place to thank your
%% supervisor and the people responsible for providing computer support ;-)
%\prefacesection{Layperson's abstract}
%An optional section suggested by the UoM thesis preparation guide.
%
\prefacesection{Acknowledgements}
%%Writing this thesis has been an exercise in sustained suffering. The casual reader may, perhaps, exempt themselves from excessive guilt, but for those of you who have played the larger role in prolonging my agonies with your encouragement and support, well…you know who you are, and I hope you get what you deserve.
%
%I wish I could write that this PhD had been an enjoyable experience, but truth be told, on the whole it has been the most depressing, demoralising, and soul-destroying experience of my life. I would have quit several years ago, but carried on obstinately out of a vague notion of duty to those who have supported and encouraged me along the way. In all honesty, I regret doing this PhD and the years I have lost from undertaking it.
%\\ \\
\noindent
This work was funded by the Natural Environment Research Council doctoral training grant number NE/L501591/1.
\\~\\
\noindent
Data was provided by the British Atmospheric Data Centre, the Environment Agency (England), Ordnance Survey Open Data, and the UK National River Flow Archives. 
\\~\\
\noindent
This work used the ARCHER UK National Supercomputing Service (\url{http://www.archer.ac.uk}) and the Centre for Environmental Data Analysis JASMIN computing facility (\url{http://www.jasmin.ac.uk/}).
% The next line is NOT optional and MUST appear
\afterpreface
